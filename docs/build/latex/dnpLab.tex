%% Generated by Sphinx.
\def\sphinxdocclass{report}
\documentclass[letterpaper,10pt,english]{sphinxmanual}
\ifdefined\pdfpxdimen
   \let\sphinxpxdimen\pdfpxdimen\else\newdimen\sphinxpxdimen
\fi \sphinxpxdimen=.75bp\relax

\PassOptionsToPackage{warn}{textcomp}
\usepackage[utf8]{inputenc}
\ifdefined\DeclareUnicodeCharacter
% support both utf8 and utf8x syntaxes
  \ifdefined\DeclareUnicodeCharacterAsOptional
    \def\sphinxDUC#1{\DeclareUnicodeCharacter{"#1}}
  \else
    \let\sphinxDUC\DeclareUnicodeCharacter
  \fi
  \sphinxDUC{00A0}{\nobreakspace}
  \sphinxDUC{2500}{\sphinxunichar{2500}}
  \sphinxDUC{2502}{\sphinxunichar{2502}}
  \sphinxDUC{2514}{\sphinxunichar{2514}}
  \sphinxDUC{251C}{\sphinxunichar{251C}}
  \sphinxDUC{2572}{\textbackslash}
\fi
\usepackage{cmap}
\usepackage[T1]{fontenc}
\usepackage{amsmath,amssymb,amstext}
\usepackage{babel}



\usepackage{times}
\expandafter\ifx\csname T@LGR\endcsname\relax
\else
% LGR was declared as font encoding
  \substitutefont{LGR}{\rmdefault}{cmr}
  \substitutefont{LGR}{\sfdefault}{cmss}
  \substitutefont{LGR}{\ttdefault}{cmtt}
\fi
\expandafter\ifx\csname T@X2\endcsname\relax
  \expandafter\ifx\csname T@T2A\endcsname\relax
  \else
  % T2A was declared as font encoding
    \substitutefont{T2A}{\rmdefault}{cmr}
    \substitutefont{T2A}{\sfdefault}{cmss}
    \substitutefont{T2A}{\ttdefault}{cmtt}
  \fi
\else
% X2 was declared as font encoding
  \substitutefont{X2}{\rmdefault}{cmr}
  \substitutefont{X2}{\sfdefault}{cmss}
  \substitutefont{X2}{\ttdefault}{cmtt}
\fi


\usepackage[Sonny]{fncychap}
\ChNameVar{\Large\normalfont\sffamily}
\ChTitleVar{\Large\normalfont\sffamily}
\usepackage{sphinx}

\fvset{fontsize=\small}
\usepackage{geometry}


% Include hyperref last.
\usepackage{hyperref}
% Fix anchor placement for figures with captions.
\usepackage{hypcap}% it must be loaded after hyperref.
% Set up styles of URL: it should be placed after hyperref.
\urlstyle{same}
\addto\captionsenglish{\renewcommand{\contentsname}{Documentation:}}

\usepackage{sphinxmessages}
\setcounter{tocdepth}{1}



\title{dnpLab Documentation}
\date{Aug 18, 2020}
\release{1.0.0}
\author{Timothy Keller}
\newcommand{\sphinxlogo}{\vbox{}}
\renewcommand{\releasename}{Release}
\makeindex
\begin{document}

\ifdefined\shorthandoff
  \ifnum\catcode`\=\string=\active\shorthandoff{=}\fi
  \ifnum\catcode`\"=\active\shorthandoff{"}\fi
\fi

\pagestyle{empty}
\sphinxmaketitle
\pagestyle{plain}
\sphinxtableofcontents
\pagestyle{normal}
\phantomsection\label{\detokenize{index::doc}}


Welcome to the dnpLab documentation. dnpLab is an Open Source Python package for importing and processing Dynamic Nuclear Polarization (DNP) data. The aim of the project is to provide a free, turn\sphinxhyphen{}key processing package for easy processing and analysis of DMP\sphinxhyphen{}NMR data.

dnpLab is a collaborative project created by
\begin{itemize}
\item {} 


\item {} 
The  at University of California, Santa Barbara

\item {} 
The  at Syracuse University

\item {} 
License: \sphinxhref{https://en.wikipedia.org/wiki/MIT\_License}{MIT License}

\end{itemize}

The source code for the project is published at: XXXX Need GitHub Link XXXX


\begin{savenotes}\sphinxattablestart
\centering
\begin{tabular}[t]{|\X{60}{100}|\X{40}{100}|}
\hline

Current Release
&
1.0.0
\\
\hline
Documentation Build Date
&
08/18/2020, 10:01:24
\\
\hline
Author(s)
&
The dnpLab Team
\\
\hline
\end{tabular}
\par
\sphinxattableend\end{savenotes}


\chapter{Features}
\label{\detokenize{index:features}}\begin{itemize}
\item {} 
Import NMR spectra in various formats (Bruker \sphinxhyphen{} TopSpin, Varian \sphinxhyphen{} (Open) VnmrJ, Magritek \sphinxhyphen{} Kea)

\item {} 
Process NMR data

\item {} 
Extract Hydration Dynamics Information

\item {} 
Create Publication Quality Figures

\end{itemize}


\section{Introduction to dnpLab}
\label{\detokenize{introduction:introduction-to-dnplab}}\label{\detokenize{introduction::doc}}
The aim of dnpLab is to provide a turn\sphinxhyphen{}key data processing environment for DNP\sphinxhyphen{}NMR data. The software package is entirely written in Python and no proprietary software is required. dnpLab is published under the open\sphinxhyphen{}source MIT license.

In the following section, we introduce the intended workflow for processing DNP\sphinxhyphen{}NMR data with dnpLab. The dnpLab python package supports data formats of all major NMR platforms.

The general workflow is as follows:
\begin{enumerate}
\sphinxsetlistlabels{\arabic}{enumi}{enumii}{}{.}%
\item {} 
Import DNP\sphinxhyphen{}NMR Data

\item {} 
Create Workspace

\item {} 
Process Data

\item {} 
Save Data in h5 Format

\item {} 
Further Processing and Analysis

\end{enumerate}

A key\sphinxhyphen{}feature of dnpLab is creating a workspace. The imported data is stored in a dnpdata object and the first object that is created during the import process is the \sphinxstyleemphasis{raw} object. It contains the raw data from the spectrometer and will be accessible at any time. All processing steps are automatically documented and the entire workspace can be saved as a single file in the h5 format.


\subsection{Workflow}
\label{\detokenize{introduction:workflow}}
\begin{figure}[htbp]
\centering
\capstart

\noindent\sphinxincludegraphics[width=400\sphinxpxdimen]{{dnpLab_workflow}.png}
\caption{Overview of the dnpLab Workflow}\label{\detokenize{introduction:id1}}\end{figure}


\subsubsection{Importing Data}
\label{\detokenize{introduction:importing-data}}
The data is imported using the {\hyperref[\detokenize{dnpImport:import}]{\sphinxcrossref{\DUrole{std,std-ref}{dnpImport}}}} sub\sphinxhyphen{}package. This sub\sphinxhyphen{}package contians modules for importing various spectrometer formats (e.g. {\hyperref[\detokenize{dnpImport:topspin}]{\sphinxcrossref{\DUrole{std,std-ref}{Topspin}}}}, {\hyperref[\detokenize{dnpImport:vnmrj}]{\sphinxcrossref{\DUrole{std,std-ref}{VnmrJ}}}}, {\hyperref[\detokenize{dnpImport:prospa}]{\sphinxcrossref{\DUrole{std,std-ref}{Prospa}}}}).

The data is imported as a {\hyperref[\detokenize{dnpData:dnpdata}]{\sphinxcrossref{\DUrole{std,std-ref}{dnpdata}}}} object. The dnpdata object is a container for data (values), coordinates for each dimension (coords), dimension labels (dims), and experimental parameters (attrs). In addition, each processing step applied to the data is saved in the dnpdata object (stored as proc\_attrs).

The dnpdata object is a flexible data format which can handle N\sphinxhyphen{}dimensional data and coordinates together.


\subsubsection{Creating a workspace}
\label{\detokenize{introduction:creating-a-workspace}}
The workspace can be created with the "create\_workspace" function in dnpLab. Once the data is imported, it is added to a workspace which is a python dictonary\sphinxhyphen{}like class that stores multiple dnpdata objects. A workspace is a collection of dnpdata objects and allows for raw and processed data to be saved in the same h5 file. That way, the raw data is always available, even if the data on the spectrometer does not exist anymore.

Creating a single h5 file has the advantage that data can be easily shared among collaborators.


\subsubsection{Processing Data}
\label{\detokenize{introduction:processing-data}}
The dnpLab workspace has the concept of a "processing\_buffer" (typically called proc). The processing buffer specifies the data which is meant for processing. Typically one will add (raw) data to the workspace and copy or move the data to the processing buffer (proc). dnpLab is primarily designed for processing and analyzing DNP\sphinxhyphen{}NMR data. Processing DNP\sphinxhyphen{}NMR data is performed using the the {\hyperref[\detokenize{dnpNMR:dnpnmr}]{\sphinxcrossref{\DUrole{std,std-ref}{dnpNMR}}}} module.


\subsubsection{Saving Data in h5 format}
\label{\detokenize{introduction:saving-data-in-h5-format}}
Once the data is processed, the entire workspace can be saved in a single file in the h5 format. This is done using the {\hyperref[\detokenize{dnpImport:h5}]{\sphinxcrossref{\DUrole{std,std-ref}{h5}}}} module. The workspace can then be loaded, subsequent processing can be performed and the data can be saved again.


\section{Installing dnpLab}
\label{\detokenize{install:installing-dnplab}}\label{\detokenize{install::doc}}

\subsection{Required Packages}
\label{\detokenize{install:required-packages}}
The following packages are required to run dnpLab:


\begin{savenotes}\sphinxattablestart
\centering
\begin{tabular}[t]{|\X{40}{100}|\X{60}{100}|}
\hline

\sphinxstylestrong{Package}
&
\sphinxstylestrong{Version}
\\
\hline
NumPy
&
1.19 or higher
\\
\hline
SciPy
&
1.5 or higher
\\
\hline
Matplotlib
&
3.3 or higher
\\
\hline
h5py
&
2.10 or higher
\\
\hline
PyQt5
&
5.15
\\
\hline
\end{tabular}
\par
\sphinxattableend\end{savenotes}


\subsection{Installing with pip}
\label{\detokenize{install:installing-with-pip}}\label{\detokenize{install:installing}}
dnpLab can be installed using pip. In a terminal simply type the following command:

\begin{sphinxVerbatim}[commandchars=\\\{\}]
\PYGZdl{} python \PYGZhy{}m pip install dnpLab
\end{sphinxVerbatim}

If you prefer to install dnpLab from the source code, check out our GitHub repository: .


\section{Quick\sphinxhyphen{}Start Guide}
\label{\detokenize{quick-start:quick-start-guide}}\label{\detokenize{quick-start::doc}}

\subsection{Importing the Package}
\label{\detokenize{quick-start:importing-the-package}}
Once the package has been {\hyperref[\detokenize{install:installing}]{\sphinxcrossref{\DUrole{std,std-ref}{installed via pip}}}}, you should be able to import dnpLab in a python terminal/script.

\begin{sphinxVerbatim}[commandchars=\\\{\}]
\PYG{k+kn}{import} \PYG{n+nn}{dnpLab} \PYG{k}{as} \PYG{n+nn}{dnp}
\end{sphinxVerbatim}


\subsection{Importing data}
\label{\detokenize{quick-start:importing-data}}
\begin{sphinxVerbatim}[commandchars=\\\{\}]
\PYG{c+c1}{\PYGZsh{} Import module}
\PYG{k+kn}{import} \PYG{n+nn}{dnpLab} \PYG{k}{as} \PYG{n+nn}{dnp}

\PYG{c+c1}{\PYGZsh{} import Topspin Data}
\PYG{n}{path} \PYG{o}{=} \PYG{l+s+s1}{\PYGZsq{}}\PYG{l+s+s1}{path/to/data}\PYG{l+s+s1}{\PYGZsq{}}
\PYG{n}{data} \PYG{o}{=} \PYG{n}{dnpLab}\PYG{o}{.}\PYG{n}{dnpImport}\PYG{o}{.}\PYG{n}{topspin}\PYG{o}{.}\PYG{n}{import\PYGZus{}topspin}\PYG{p}{(}\PYG{n}{path}\PYG{p}{)}

\PYG{c+c1}{\PYGZsh{} create workspace for processing data}
\PYG{n}{workspace} \PYG{o}{=} \PYG{n}{dnp}\PYG{o}{.}\PYG{n}{create\PYGZus{}workspace}\PYG{p}{(}\PYG{l+s+s1}{\PYGZsq{}}\PYG{l+s+s1}{raw}\PYG{l+s+s1}{\PYGZsq{}}\PYG{p}{,} \PYG{n}{data}\PYG{p}{)}
\end{sphinxVerbatim}


\subsection{Processing NMR Data}
\label{\detokenize{quick-start:processing-nmr-data}}
\begin{sphinxVerbatim}[commandchars=\\\{\}]
\PYG{c+c1}{\PYGZsh{} Remove DC offset from FID}
\PYG{n}{workspace} \PYG{o}{=} \PYG{n}{dnpLab}\PYG{o}{.}\PYG{n}{dnpNMR}\PYG{o}{.}\PYG{n}{remove\PYGZus{}offset}\PYG{p}{(}\PYG{n}{workspace}\PYG{p}{,} \PYG{p}{\PYGZob{}}\PYG{p}{\PYGZcb{}}\PYG{p}{)}
\PYG{c+c1}{\PYGZsh{} Apply Exponential Apodization to data}
\PYG{n}{workspace} \PYG{o}{=} \PYG{n}{dnpLab}\PYG{o}{.}\PYG{n}{dnpNMR}\PYG{o}{.}\PYG{n}{window}\PYG{p}{(}\PYG{n}{workspace}\PYG{p}{,} \PYG{p}{\PYGZob{}}\PYG{p}{\PYGZcb{}}\PYG{p}{)}
\PYG{c+c1}{\PYGZsh{} Apply Fourier Transform to direct dimension by default (t2)}
\PYG{n}{workspace} \PYG{o}{=} \PYG{n}{dnpLab}\PYG{o}{.}\PYG{n}{dnpNMR}\PYG{o}{.}\PYG{n}{fourier\PYGZus{}transform}\PYG{p}{(}\PYG{n}{workspace}\PYG{p}{,} \PYG{p}{\PYGZob{}}\PYG{p}{\PYGZcb{}}\PYG{p}{)}
\end{sphinxVerbatim}


\subsection{Example Script}
\label{\detokenize{quick-start:example-script}}

\section{dnpLab Examples}
\label{\detokenize{examples:dnplab-examples}}\label{\detokenize{examples::doc}}
dnpLab comes with many example scripts to demonstrate how the package can be used to import data from different spectrometer platform, process NMR data and extract enhancement data or hydration information. The example scripts are located in the \sphinxstyleemphasis{examples} folder using sample data located in the \sphinxstyleemphasis{data} folder.

If you installed dnpLab using pip you can download the example scripts and data from the GitHub repository:

Example Scripts:    \sphinxhref{http://link}{Example Scripts}
Example Data:    \sphinxhref{http://link}{Example Data}


\subsection{Import Data and Process FID (Bruker Format)}
\label{\detokenize{examples:import-data-and-process-fid-bruker-format}}
This example uses the example script: \sphinxstyleemphasis{example\_process\_1Dbruker.py}. The script demonstrates the following features of dnpLab:
\begin{enumerate}
\sphinxsetlistlabels{\arabic}{enumi}{enumii}{}{.}%
\item {} 
Load a single FID (Bruker format)

\item {} 
Perform an offset correction

\item {} 
Apply apodization to the FID

\item {} 
Perform a Fourier transformation

\item {} 
Phase correct the resulting spectrum

\end{enumerate}

If you installed dnpLab using pip. Otherwise, you have to specify the path to the package explicitly:

\begin{sphinxVerbatim}[commandchars=\\\{\}]
\PYG{k+kn}{import} \PYG{n+nn}{numpy} \PYG{k}{as} \PYG{n+nn}{np}
\PYG{k+kn}{import} \PYG{n+nn}{dnpLab} \PYG{k}{as} \PYG{n+nn}{dnp}
\end{sphinxVerbatim}

\begin{sphinxadmonition}{note}{Note:}
If you downloaded dnpLab via GitHub and haven\textquotesingle{}t installed, you must add the directory for dnpLab to the system path before importing dnpLab. Add the following lines to the beginning of the script:

\begin{sphinxVerbatim}[commandchars=\\\{\}]
\PYG{k+kn}{import} \PYG{n+nn}{sys}
\PYG{n}{sys}\PYG{o}{.}\PYG{n}{path}\PYG{o}{.}\PYG{n}{append}\PYG{p}{(}\PYG{l+s+s1}{\PYGZsq{}}\PYG{l+s+s1}{path/to/dnpLab/package}\PYG{l+s+s1}{\PYGZsq{}}\PYG{p}{)}
\end{sphinxVerbatim}
\end{sphinxadmonition}

In the next step load a single FID in Bruker format:

\begin{sphinxVerbatim}[commandchars=\\\{\}]
\PYG{n}{path} \PYG{o}{=} \PYG{l+s+s1}{\PYGZsq{}}\PYG{l+s+s1}{path/to/data/topspin/}\PYG{l+s+s1}{\PYGZsq{}}
\PYG{n}{folder} \PYG{o}{=} \PYG{l+m+mi}{20}

\PYG{n}{data} \PYG{o}{=} \PYG{n}{data}\PYG{o}{.}\PYG{n}{dnpImport}\PYG{o}{.}\PYG{n}{topspin}\PYG{o}{.}\PYG{n}{import\PYGZus{}topspin}\PYG{p}{(}\PYG{n}{path}\PYG{p}{,}\PYG{n}{folder}\PYG{p}{)}
\end{sphinxVerbatim}

The topspin import module requires the path and the folder number.
In the next step the workspace is set up and the imported data is added to the \sphinxstyleemphasis{raw} workspace and the same data is copied to the \sphinxstyleemphasis{proc} workspace.

\begin{sphinxVerbatim}[commandchars=\\\{\}]
\PYG{n}{ws} \PYG{o}{=} \PYG{n}{dnp}\PYG{o}{.}\PYG{n}{create\PYGZus{}workspace}\PYG{p}{(}\PYG{p}{)}
\PYG{n}{ws}\PYG{o}{.}\PYG{n}{add}\PYG{p}{(}\PYG{l+s+s1}{\PYGZsq{}}\PYG{l+s+s1}{raw}\PYG{l+s+s1}{\PYGZsq{}}\PYG{p}{,} \PYG{n}{data}\PYG{p}{)}
\PYG{n}{ws}\PYG{o}{.}\PYG{n}{copy}\PYG{p}{(}\PYG{l+s+s1}{\PYGZsq{}}\PYG{l+s+s1}{raw}\PYG{l+s+s1}{\PYGZsq{}}\PYG{p}{,} \PYG{l+s+s1}{\PYGZsq{}}\PYG{l+s+s1}{proc}\PYG{l+s+s1}{\PYGZsq{}}\PYG{p}{)}
\end{sphinxVerbatim}

\begin{sphinxadmonition}{note}{Note:}
When working with dnpLab one of the first steps is to copy the imported data to the \sphinxstyleemphasis{raw} workspace. That way the raw data and all it\textquotesingle{}s attributes will be always accessible to the user. When saving data with dnpLab the raw data is safed toegether with the processed data. dnpLab uses the h5 format to store data.
\end{sphinxadmonition}

In the following steps, the FID is processed and the spectrum is plotted.

\begin{sphinxVerbatim}[commandchars=\\\{\}]
\PYG{n}{dnp}\PYG{o}{.}\PYG{n}{dnpNMR}\PYG{o}{.}\PYG{n}{remove\PYGZus{}offset}\PYG{p}{(}\PYG{n}{ws}\PYG{p}{,}\PYG{p}{\PYGZob{}}\PYG{p}{\PYGZcb{}}\PYG{p}{)}
\PYG{n}{dnp}\PYG{o}{.}\PYG{n}{dnpNMR}\PYG{o}{.}\PYG{n}{window}\PYG{p}{(}\PYG{n}{ws}\PYG{p}{,}\PYG{p}{\PYGZob{}}\PYG{l+s+s1}{\PYGZsq{}}\PYG{l+s+s1}{linewidth}\PYG{l+s+s1}{\PYGZsq{}} \PYG{p}{:} \PYG{l+m+mi}{10}\PYG{p}{\PYGZcb{}}\PYG{p}{)}
\PYG{n}{dnp}\PYG{o}{.}\PYG{n}{dnpNMR}\PYG{o}{.}\PYG{n}{fourier\PYGZus{}transform}\PYG{p}{(}\PYG{n}{ws}\PYG{p}{,}\PYG{p}{\PYGZob{}}\PYG{l+s+s1}{\PYGZsq{}}\PYG{l+s+s1}{zero\PYGZus{}fill\PYGZus{}factor}\PYG{l+s+s1}{\PYGZsq{}} \PYG{p}{:} \PYG{l+m+mi}{2}\PYG{p}{\PYGZcb{}}\PYG{p}{)}
\PYG{n}{dnp}\PYG{o}{.}\PYG{n}{dnpNMR}\PYG{o}{.}\PYG{n}{autophase}\PYG{p}{(}\PYG{n}{ws}\PYG{p}{,}\PYG{p}{\PYGZob{}}\PYG{p}{\PYGZcb{}}\PYG{p}{)}
\end{sphinxVerbatim}

In this example first a baseline correction is performed (dnpNMR.remove\_offset) and apodization is applied ot the FID (dnpNMR.window). In this example a line broadening of 10 Hz is applied. The next step is to Fourier transform the FID (dnpNMR.fourier\_transform) and phase the spectrum (dnpNMR.autophase).

To plot the NMR spectrum:

\begin{sphinxVerbatim}[commandchars=\\\{\}]
\PYG{n}{dnp}\PYG{o}{.}\PYG{n}{dnpResults}\PYG{o}{.}\PYG{n}{figure}\PYG{p}{(}\PYG{p}{)}
\PYG{n}{dnp}\PYG{o}{.}\PYG{n}{dnpResults}\PYG{o}{.}\PYG{n}{plot}\PYG{p}{(}\PYG{n}{ws}\PYG{p}{[}\PYG{l+s+s1}{\PYGZsq{}}\PYG{l+s+s1}{proc}\PYG{l+s+s1}{\PYGZsq{}}\PYG{p}{]}\PYG{o}{.}\PYG{n}{real}\PYG{p}{)}
\PYG{n}{dnp}\PYG{o}{.}\PYG{n}{dnpResults}\PYG{o}{.}\PYG{n}{xlim}\PYG{p}{(}\PYG{p}{[}\PYG{o}{\PYGZhy{}}\PYG{l+m+mi}{35}\PYG{p}{,}\PYG{l+m+mi}{50}\PYG{p}{]}\PYG{p}{)}
\PYG{n}{dnp}\PYG{o}{.}\PYG{n}{dnpResults}\PYG{o}{.}\PYG{n}{plt}\PYG{o}{.}\PYG{n}{xlabel}\PYG{p}{(}\PYG{l+s+s1}{\PYGZsq{}}\PYG{l+s+s1}{Chemical Shift [ppm]}\PYG{l+s+s1}{\PYGZsq{}}\PYG{p}{)}
\PYG{n}{dnp}\PYG{o}{.}\PYG{n}{dnpResults}\PYG{o}{.}\PYG{n}{plt}\PYG{o}{.}\PYG{n}{ylabel}\PYG{p}{(}\PYG{l+s+s1}{\PYGZsq{}}\PYG{l+s+s1}{Signal Amplitude [a.u.]}\PYG{l+s+s1}{\PYGZsq{}}\PYG{p}{)}
\PYG{n}{dnp}\PYG{o}{.}\PYG{n}{dnpResults}\PYG{o}{.}\PYG{n}{show}\PYG{p}{(}\PYG{p}{)}
\end{sphinxVerbatim}

\begin{figure}[htbp]
\centering
\capstart

\noindent\sphinxincludegraphics[width=400\sphinxpxdimen]{{example_process_1dbruker_real}.png}
\caption{1D NMR Spectrum Imported in Bruker Format}\label{\detokenize{examples:id1}}\label{\detokenize{examples:index-1dbrukerreal}}\end{figure}

Here only the real part of the spectrum is displayed (dnpResults.plot(ws{[}\textquotesingle{}proc\textquotesingle{}{]}.real)). The imaginary part of the spectrum can be displayed by changing the second line to

\begin{sphinxVerbatim}[commandchars=\\\{\}]
\PYG{n}{dnpResults}\PYG{o}{.}\PYG{n}{plot}\PYG{p}{(}\PYG{n}{ws}\PYG{p}{[}\PYG{l+s+s1}{\PYGZsq{}}\PYG{l+s+s1}{proc}\PYG{l+s+s1}{\PYGZsq{}}\PYG{p}{]}\PYG{o}{.}\PYG{n}{imag}\PYG{p}{)}
\end{sphinxVerbatim}

To display the unprocessed raw FID:

\begin{sphinxVerbatim}[commandchars=\\\{\}]
\PYG{n}{dnp}\PYG{o}{.}\PYG{n}{dnpResults}\PYG{o}{.}\PYG{n}{figure}\PYG{p}{(}\PYG{p}{)}
\PYG{n}{dnp}\PYG{o}{.}\PYG{n}{dnpResults}\PYG{o}{.}\PYG{n}{plot}\PYG{p}{(}\PYG{n}{ws}\PYG{p}{[}\PYG{l+s+s1}{\PYGZsq{}}\PYG{l+s+s1}{raw}\PYG{l+s+s1}{\PYGZsq{}}\PYG{p}{]}\PYG{o}{.}\PYG{n}{real}\PYG{p}{)}
\PYG{n}{dnp}\PYG{o}{.}\PYG{n}{dnpResults}\PYG{o}{.}\PYG{n}{plt}\PYG{o}{.}\PYG{n}{xlabel}\PYG{p}{(}\PYG{l+s+s1}{\PYGZsq{}}\PYG{l+s+s1}{t2 [s]}\PYG{l+s+s1}{\PYGZsq{}}\PYG{p}{)}
\PYG{n}{dnp}\PYG{o}{.}\PYG{n}{dnpResults}\PYG{o}{.}\PYG{n}{plt}\PYG{o}{.}\PYG{n}{ylabel}\PYG{p}{(}\PYG{l+s+s1}{\PYGZsq{}}\PYG{l+s+s1}{Signal Amplitude [a.u.]}\PYG{l+s+s1}{\PYGZsq{}}\PYG{p}{)}
\PYG{n}{dnp}\PYG{o}{.}\PYG{n}{dnpResults}\PYG{o}{.}\PYG{n}{show}\PYG{p}{(}\PYG{p}{)}
\end{sphinxVerbatim}

\begin{figure}[htbp]
\centering
\capstart

\noindent\sphinxincludegraphics[width=400\sphinxpxdimen]{{example_FID_1dbruker_real}.png}
\caption{1D FID from raw data (Bruker Format)}\label{\detokenize{examples:id2}}\label{\detokenize{examples:index-1dfidbrukerreal}}\end{figure}


\subsection{Determine T1 from an Inversion Recovery Experiment}
\label{\detokenize{examples:determine-t1-from-an-inversion-recovery-experiment}}
In this example, the data from an inversion recovery experiment is analyzed to extract the longitudinal relaxation time T1 from the polarization build up. This example uses the example script: \sphinxstyleemphasis{example\_process\_IRbruker.py}.

First, import the experimental data (Bruker format) (if dnpLab is installed through pip, ignore the first two lines):

\begin{sphinxVerbatim}[commandchars=\\\{\}]
\PYG{k+kn}{import} \PYG{n+nn}{sys}
\PYG{n}{sys}\PYG{o}{.}\PYG{n}{path}\PYG{o}{.}\PYG{n}{append}\PYG{p}{(}\PYG{l+s+s1}{\PYGZsq{}}\PYG{l+s+s1}{path/to/dnpLab/package}\PYG{l+s+s1}{\PYGZsq{}}\PYG{p}{)}

\PYG{k+kn}{import} \PYG{n+nn}{numpy} \PYG{k}{as} \PYG{n+nn}{np}
\PYG{k+kn}{import} \PYG{n+nn}{dnpLab} \PYG{k}{as} \PYG{n+nn}{dnp}
\end{sphinxVerbatim}

In the next step load a single FID in Bruker format:

\begin{sphinxVerbatim}[commandchars=\\\{\}]
\PYG{n}{path} \PYG{o}{=} \PYG{l+s+s1}{\PYGZsq{}}\PYG{l+s+s1}{path/to/data/topspin/}\PYG{l+s+s1}{\PYGZsq{}}
\PYG{n}{folder} \PYG{o}{=} \PYG{l+m+mi}{304}

\PYG{n}{data} \PYG{o}{=} \PYG{n}{dnp}\PYG{o}{.}\PYG{n}{dnpImport}\PYG{o}{.}\PYG{n}{topspin}\PYG{o}{.}\PYG{n}{import\PYGZus{}topspin}\PYG{p}{(}\PYG{n}{path}\PYG{p}{,}\PYG{n}{folder}\PYG{p}{)}
\end{sphinxVerbatim}

Next, create the workspace:

\begin{sphinxVerbatim}[commandchars=\\\{\}]
\PYG{n}{ws} \PYG{o}{=} \PYG{n}{dnp}\PYG{o}{.}\PYG{n}{create\PYGZus{}workspace}\PYG{p}{(}\PYG{p}{)}
\PYG{n}{ws}\PYG{o}{.}\PYG{n}{add}\PYG{p}{(}\PYG{l+s+s1}{\PYGZsq{}}\PYG{l+s+s1}{raw}\PYG{l+s+s1}{\PYGZsq{}}\PYG{p}{,} \PYG{n}{data}\PYG{p}{)}
\PYG{n}{ws}\PYG{o}{.}\PYG{n}{copy}\PYG{p}{(}\PYG{l+s+s1}{\PYGZsq{}}\PYG{l+s+s1}{raw}\PYG{l+s+s1}{\PYGZsq{}}\PYG{p}{,} \PYG{l+s+s1}{\PYGZsq{}}\PYG{l+s+s1}{proc}\PYG{l+s+s1}{\PYGZsq{}}\PYG{p}{)}
\end{sphinxVerbatim}

Next, process the FID, perform Fourier transformation, align and phase the NMR spectra:

\begin{sphinxVerbatim}[commandchars=\\\{\}]
\PYG{n}{dnp}\PYG{o}{.}\PYG{n}{dnpNMR}\PYG{o}{.}\PYG{n}{remove\PYGZus{}offset}\PYG{p}{(}\PYG{n}{ws}\PYG{p}{,}\PYG{p}{\PYGZob{}}\PYG{p}{\PYGZcb{}}\PYG{p}{)}
\PYG{n}{dnp}\PYG{o}{.}\PYG{n}{dnpNMR}\PYG{o}{.}\PYG{n}{window}\PYG{p}{(}\PYG{n}{ws}\PYG{p}{,}\PYG{p}{\PYGZob{}}\PYG{l+s+s1}{\PYGZsq{}}\PYG{l+s+s1}{linewidth}\PYG{l+s+s1}{\PYGZsq{}} \PYG{p}{:} \PYG{l+m+mi}{10}\PYG{p}{\PYGZcb{}}\PYG{p}{)}
\PYG{n}{dnp}\PYG{o}{.}\PYG{n}{dnpNMR}\PYG{o}{.}\PYG{n}{fourier\PYGZus{}transform}\PYG{p}{(}\PYG{n}{ws}\PYG{p}{,}\PYG{p}{\PYGZob{}}\PYG{l+s+s1}{\PYGZsq{}}\PYG{l+s+s1}{zero\PYGZus{}fill\PYGZus{}factor}\PYG{l+s+s1}{\PYGZsq{}} \PYG{p}{:} \PYG{l+m+mi}{2}\PYG{p}{\PYGZcb{}}\PYG{p}{)}
\PYG{n}{dnp}\PYG{o}{.}\PYG{n}{dnpNMR}\PYG{o}{.}\PYG{n}{align}\PYG{p}{(}\PYG{n}{ws}\PYG{p}{,} \PYG{p}{\PYGZob{}}\PYG{p}{\PYGZcb{}}\PYG{p}{)}
\PYG{n}{dnp}\PYG{o}{.}\PYG{n}{dnpNMR}\PYG{o}{.}\PYG{n}{autophase}\PYG{p}{(}\PYG{n}{ws}\PYG{p}{,}\PYG{p}{\PYGZob{}}\PYG{p}{\PYGZcb{}}\PYG{p}{)}
\end{sphinxVerbatim}

To plot the processed NMR spectra:

\begin{sphinxVerbatim}[commandchars=\\\{\}]
\PYG{n}{dnp}\PYG{o}{.}\PYG{n}{dnpResults}\PYG{o}{.}\PYG{n}{plot}\PYG{p}{(}\PYG{n}{ws}\PYG{p}{[}\PYG{l+s+s1}{\PYGZsq{}}\PYG{l+s+s1}{ft}\PYG{l+s+s1}{\PYGZsq{}}\PYG{p}{]}\PYG{o}{.}\PYG{n}{real}\PYG{p}{)}
\PYG{n}{dnp}\PYG{o}{.}\PYG{n}{dnpResults}\PYG{o}{.}\PYG{n}{xlim}\PYG{p}{(}\PYG{p}{[}\PYG{o}{\PYGZhy{}}\PYG{l+m+mi}{30}\PYG{p}{,}\PYG{l+m+mi}{50}\PYG{p}{]}\PYG{p}{)}
\PYG{n}{dnp}\PYG{o}{.}\PYG{n}{dnpResults}\PYG{o}{.}\PYG{n}{plt}\PYG{o}{.}\PYG{n}{xlabel}\PYG{p}{(}\PYG{l+s+s1}{\PYGZsq{}}\PYG{l+s+s1}{Chemical Shift [ppm]}\PYG{l+s+s1}{\PYGZsq{}}\PYG{p}{)}
\PYG{n}{dnp}\PYG{o}{.}\PYG{n}{dnpResults}\PYG{o}{.}\PYG{n}{plt}\PYG{o}{.}\PYG{n}{ylabel}\PYG{p}{(}\PYG{l+s+s1}{\PYGZsq{}}\PYG{l+s+s1}{Signal Amplitude [a.u.]}\PYG{l+s+s1}{\PYGZsq{}}\PYG{p}{)}
\PYG{n}{dnp}\PYG{o}{.}\PYG{n}{dnpResults}\PYG{o}{.}\PYG{n}{figure}\PYG{p}{(}\PYG{p}{)}
\end{sphinxVerbatim}

\begin{figure}[htbp]
\centering
\capstart

\noindent\sphinxincludegraphics[width=400\sphinxpxdimen]{{example_process_IRbruker}.png}
\caption{Processed inversion recovery spectra (Bruker Format)}\label{\detokenize{examples:id3}}\label{\detokenize{examples:index-irbruker}}\end{figure}

Next, the processed NMR spectra are copied to \sphinxstyleemphasis{ft} within the workspace, the signal amplitude for each NMR spectrum is integrated and the data is fitted to a function, describing inversion recovery polarization build\sphinxhyphen{}up.

\begin{sphinxVerbatim}[commandchars=\\\{\}]
\PYG{n}{ws}\PYG{o}{.}\PYG{n}{copy}\PYG{p}{(}\PYG{l+s+s1}{\PYGZsq{}}\PYG{l+s+s1}{proc}\PYG{l+s+s1}{\PYGZsq{}}\PYG{p}{,} \PYG{l+s+s1}{\PYGZsq{}}\PYG{l+s+s1}{ft}\PYG{l+s+s1}{\PYGZsq{}}\PYG{p}{)}
\PYG{n}{dnp}\PYG{o}{.}\PYG{n}{dnpNMR}\PYG{o}{.}\PYG{n}{integrate}\PYG{p}{(}\PYG{n}{ws}\PYG{p}{,} \PYG{p}{\PYGZob{}}\PYG{l+s+s1}{\PYGZsq{}}\PYG{l+s+s1}{integrate\PYGZus{}width}\PYG{l+s+s1}{\PYGZsq{}} \PYG{p}{:} \PYG{l+m+mi}{100}\PYG{p}{,} \PYG{l+s+s1}{\PYGZsq{}}\PYG{l+s+s1}{integrate\PYGZus{}center}\PYG{l+s+s1}{\PYGZsq{}} \PYG{p}{:} \PYG{l+m+mi}{0}\PYG{p}{\PYGZcb{}}\PYG{p}{)}
\PYG{n}{dnp}\PYG{o}{.}\PYG{n}{dnpFit}\PYG{o}{.}\PYG{n}{t1Fit}\PYG{p}{(}\PYG{n}{ws}\PYG{p}{)}
\end{sphinxVerbatim}

The T1 value can be displayed using:

\begin{sphinxVerbatim}[commandchars=\\\{\}]
\PYG{n+nb}{print}\PYG{p}{(}\PYG{l+s+s1}{\PYGZsq{}}\PYG{l+s+s1}{T1 value (sec) = }\PYG{l+s+s1}{\PYGZsq{}} \PYG{o}{+} \PYG{n+nb}{str}\PYG{p}{(}\PYG{n}{ws}\PYG{p}{[}\PYG{l+s+s1}{\PYGZsq{}}\PYG{l+s+s1}{fit}\PYG{l+s+s1}{\PYGZsq{}}\PYG{p}{]}\PYG{o}{.}\PYG{n}{attrs}\PYG{p}{[}\PYG{l+s+s1}{\PYGZsq{}}\PYG{l+s+s1}{t1}\PYG{l+s+s1}{\PYGZsq{}}\PYG{p}{]}\PYG{p}{)}\PYG{p}{)}
\PYG{n}{T1} \PYG{n}{value} \PYG{p}{(}\PYG{n}{sec}\PYG{p}{)} \PYG{o}{=} \PYG{l+m+mf}{2.045498109768188}
\end{sphinxVerbatim}

To plot the inversion\sphinxhyphen{}recovery build\sphinxhyphen{}up curve (experimental and fitted data):

\begin{sphinxVerbatim}[commandchars=\\\{\}]
\PYG{n}{dnp}\PYG{o}{.}\PYG{n}{dnpResults}\PYG{o}{.}\PYG{n}{plot}\PYG{p}{(}\PYG{n}{ws}\PYG{p}{[}\PYG{l+s+s1}{\PYGZsq{}}\PYG{l+s+s1}{proc}\PYG{l+s+s1}{\PYGZsq{}}\PYG{p}{]}\PYG{o}{.}\PYG{n}{real}\PYG{p}{,} \PYG{l+s+s1}{\PYGZsq{}}\PYG{l+s+s1}{o}\PYG{l+s+s1}{\PYGZsq{}}\PYG{p}{)}
\PYG{n}{dnp}\PYG{o}{.}\PYG{n}{dnpResults}\PYG{o}{.}\PYG{n}{plot}\PYG{p}{(}\PYG{n}{ws}\PYG{p}{[}\PYG{l+s+s1}{\PYGZsq{}}\PYG{l+s+s1}{fit}\PYG{l+s+s1}{\PYGZsq{}}\PYG{p}{]}\PYG{p}{)}
\PYG{n}{dnp}\PYG{o}{.}\PYG{n}{dnpResults}\PYG{o}{.}\PYG{n}{show}\PYG{p}{(}\PYG{p}{)}
\end{sphinxVerbatim}

\begin{figure}[htbp]
\centering
\capstart

\noindent\sphinxincludegraphics[width=400\sphinxpxdimen]{{example_process_IRbuildup}.png}
\caption{Inversion recovery build\sphinxhyphen{}up (experimental and fit)}\label{\detokenize{examples:id4}}\label{\detokenize{examples:index-irbuildup}}\end{figure}


\section{dnpData}
\label{\detokenize{dnpData:dnpdata}}\label{\detokenize{dnpData:id1}}\label{\detokenize{dnpData::doc}}

\subsection{dnpdata Class Overview}
\label{\detokenize{dnpData:dnpdata-class-overview}}
The dnpdata class is a flexible data container for N\sphinxhyphen{}dimensional data. The dnpdata class stores data, axes, parameters and processing information in a single object.


\subsection{dnpdata Attributes}
\label{\detokenize{dnpData:dnpdata-attributes}}
The attributes in the dnpdata object are named according to the convention by Pandas and xarray.


\begin{savenotes}\sphinxattablestart
\centering
\begin{tabulary}{\linewidth}[t]{|T|T|T|}
\hline

\sphinxstylestrong{attribute}
&
\sphinxstylestrong{type}
&
\sphinxstylestrong{description}
\\
\hline
values
&
numpy.ndarray
&
Numpy array of data values
\\
\hline
dims
&
list of str
&
Names for each of the N\sphinxhyphen{}dimensions in values
\\
\hline
coords
&
list of numpy.ndarray
&
Axes values for each of the N\sphinxhyphen{}dimensions
\\
\hline
attrs
&
dict
&
Dictionary of miscellaneous
\\
\hline
proc\_attrs
&
list of tuples (str, dict)
&
List which stores each processing step
\\
\hline
\end{tabulary}
\par
\sphinxattableend\end{savenotes}


\subsection{dnpdata Examples}
\label{\detokenize{dnpData:dnpdata-examples}}
A dnpdata object can be defined as follows:

\begin{sphinxVerbatim}[commandchars=\\\{\}]
\PYG{k+kn}{import} \PYG{n+nn}{dnpLab} \PYG{k}{as} \PYG{n+nn}{dnp}
\PYG{k+kn}{import} \PYG{n+nn}{numpy} \PYG{k}{as} \PYG{n+nn}{np}

\PYG{n}{x} \PYG{o}{=} \PYG{n}{np}\PYG{o}{.}\PYG{n}{r\PYGZus{}}\PYG{p}{[}\PYG{o}{\PYGZhy{}}\PYG{l+m+mi}{10}\PYG{p}{:}\PYG{l+m+mi}{10}\PYG{p}{:}\PYG{l+m+mi}{100}\PYG{n}{j}\PYG{p}{]}
\PYG{n}{values} \PYG{o}{=} \PYG{n}{x}\PYG{o}{*}\PYG{o}{*}\PYG{l+m+mf}{2.}
\PYG{n}{coords} \PYG{o}{=} \PYG{p}{[}\PYG{n}{x}\PYG{p}{]}
\PYG{n}{dims} \PYG{o}{=} \PYG{p}{[}\PYG{l+s+s1}{\PYGZsq{}}\PYG{l+s+s1}{x}\PYG{l+s+s1}{\PYGZsq{}}\PYG{p}{]}

\PYG{n}{data} \PYG{o}{=} \PYG{n}{dnp}\PYG{o}{.}\PYG{n}{dnpdata}\PYG{p}{(}\PYG{n}{values}\PYG{p}{,} \PYG{n}{coords}\PYG{p}{,} \PYG{n}{dims}\PYG{p}{)}

\PYG{n+nb}{print}\PYG{p}{(}\PYG{n}{data}\PYG{p}{)}
\end{sphinxVerbatim}

The dnpdata class has a number of methods for manipulating the data.

The dimensions can be renamed:

\begin{sphinxVerbatim}[commandchars=\\\{\}]
\PYG{n}{data}\PYG{o}{.}\PYG{n}{rename}\PYG{p}{(}\PYG{l+s+s1}{\PYGZsq{}}\PYG{l+s+s1}{x}\PYG{l+s+s1}{\PYGZsq{}}\PYG{p}{,} \PYG{l+s+s1}{\PYGZsq{}}\PYG{l+s+s1}{t}\PYG{l+s+s1}{\PYGZsq{}}\PYG{p}{)}
\end{sphinxVerbatim}

A N\sphinxhyphen{}dimensional data set can be reshaped by it\textquotesingle{}s dimension labels. If a data set has

\begin{sphinxVerbatim}[commandchars=\\\{\}]
\PYG{n}{data}\PYG{o}{.}\PYG{n}{reorder}\PYG{p}{(}\PYG{p}{[}\PYG{l+s+s1}{\PYGZsq{}}\PYG{l+s+s1}{z}\PYG{l+s+s1}{\PYGZsq{}}\PYG{p}{,} \PYG{l+s+s1}{\PYGZsq{}}\PYG{l+s+s1}{x}\PYG{l+s+s1}{\PYGZsq{}}\PYG{p}{,} \PYG{l+s+s1}{\PYGZsq{}}\PYG{l+s+s1}{y}\PYG{l+s+s1}{\PYGZsq{}}\PYG{p}{]}\PYG{p}{)}
\end{sphinxVerbatim}


\subsection{Indexing}
\label{\detokenize{dnpData:indexing}}
A number of methods can be used to index the data based on the coordinates.

\begin{sphinxVerbatim}[commandchars=\\\{\}]
\PYG{n}{data\PYGZus{}slice} \PYG{o}{=} \PYG{n}{data}\PYG{p}{[}\PYG{l+s+s1}{\PYGZsq{}}\PYG{l+s+s1}{t}\PYG{l+s+s1}{\PYGZsq{}}\PYG{p}{,} \PYG{l+m+mi}{0}\PYG{p}{:}\PYG{l+m+mi}{10}\PYG{p}{]} \PYG{c+c1}{\PYGZsh{} return first 10 points of data down \PYGZsq{}t\PYGZsq{} dimension}

\PYG{n}{data\PYGZus{}slice} \PYG{o}{=} \PYG{n}{data}\PYG{p}{[}\PYG{l+s+s1}{\PYGZsq{}}\PYG{l+s+s1}{t}\PYG{l+s+s1}{\PYGZsq{}}\PYG{p}{,} \PYG{p}{(}\PYG{l+m+mf}{0.1}\PYG{p}{,} \PYG{l+m+mf}{0.5}\PYG{p}{)}\PYG{p}{]} \PYG{c+c1}{\PYGZsh{} return data in range from 0.1 to 0.5}

\PYG{n}{data\PYGZus{}slice} \PYG{o}{=} \PYG{n}{data}\PYG{p}{[}\PYG{l+s+s1}{\PYGZsq{}}\PYG{l+s+s1}{t}\PYG{l+s+s1}{\PYGZsq{}}\PYG{p}{,} \PYG{l+m+mf}{0.1}\PYG{p}{]} \PYG{c+c1}{\PYGZsh{} for single float, return data at index nearest to 0.1 in time coordinates}

\PYG{n}{data\PYGZus{}slice} \PYG{o}{=} \PYG{n}{data}\PYG{p}{[}\PYG{l+s+s1}{\PYGZsq{}}\PYG{l+s+s1}{t}\PYG{l+s+s1}{\PYGZsq{}}\PYG{p}{,} \PYG{l+m+mi}{10}\PYG{p}{]} \PYG{c+c1}{\PYGZsh{} for single int, return data at index 10}
\end{sphinxVerbatim}


\subsection{dnpdata Methods}
\label{\detokenize{dnpData:dnpdata-methods}}\index{dnpdata (class in dnpLab)@\spxentry{dnpdata}\spxextra{class in dnpLab}}

\begin{fulllineitems}
\phantomsection\label{\detokenize{dnpData:dnpLab.dnpdata}}\pysiglinewithargsret{\sphinxbfcode{\sphinxupquote{class }}\sphinxcode{\sphinxupquote{dnpLab.}}\sphinxbfcode{\sphinxupquote{dnpdata}}}{\emph{\DUrole{n}{values}\DUrole{o}{=}\DUrole{default_value}{array({[}{]}, dtype=float64)}}, \emph{\DUrole{n}{coords}\DUrole{o}{=}\DUrole{default_value}{{[}{]}}}, \emph{\DUrole{n}{dims}\DUrole{o}{=}\DUrole{default_value}{{[}{]}}}, \emph{\DUrole{n}{attrs}\DUrole{o}{=}\DUrole{default_value}{\{\}}}, \emph{\DUrole{n}{procList}\DUrole{o}{=}\DUrole{default_value}{{[}{]}}}}{}
Bases: \sphinxcode{\sphinxupquote{dnpLab.core.nddata.nddata\_core}}

dnpdata Class for handling dnp data

The dnpdata class is inspired by pyspecdata nddata object which handles n\sphinxhyphen{}dimensional data, axes, and other relevant information together.

This class is designed to handle data and axes together so that performing NMR processing can be performed easily.

Attributes:
values (numpy.ndarray): Numpy Array containing data
coords (list): List of numpy arrays containing axes of data
dims (list): List of axes labels for data
attrs (dict): Dictionary of parameters for data
\index{add\_proc\_attrs() (dnpLab.dnpdata method)@\spxentry{add\_proc\_attrs()}\spxextra{dnpLab.dnpdata method}}

\begin{fulllineitems}
\phantomsection\label{\detokenize{dnpData:dnpLab.dnpdata.add_proc_attrs}}\pysiglinewithargsret{\sphinxbfcode{\sphinxupquote{add\_proc\_attrs}}}{\emph{\DUrole{n}{proc\_attr\_name}}, \emph{\DUrole{n}{proc\_dict}}}{}
Stamp processing step to dnpdata object
\begin{quote}\begin{description}
\item[{Parameters}] \leavevmode\begin{itemize}
\item {} 
\sphinxstyleliteralstrong{\sphinxupquote{proc\_attr\_name}} (\sphinxstyleliteralemphasis{\sphinxupquote{str}}) \sphinxhyphen{}\sphinxhyphen{} Name of processing step (e.g. "fourier\_transform"

\item {} 
\sphinxstyleliteralstrong{\sphinxupquote{proc\_dict}} (\sphinxstyleliteralemphasis{\sphinxupquote{dict}}) \sphinxhyphen{}\sphinxhyphen{} Dictionary of processing parameters for this processing step.

\end{itemize}

\end{description}\end{quote}

\end{fulllineitems}

\index{autophase() (dnpLab.dnpdata method)@\spxentry{autophase()}\spxextra{dnpLab.dnpdata method}}

\begin{fulllineitems}
\phantomsection\label{\detokenize{dnpData:dnpLab.dnpdata.autophase}}\pysiglinewithargsret{\sphinxbfcode{\sphinxupquote{autophase}}}{}{}
Multiply dnpdata object by phase

\end{fulllineitems}

\index{phase() (dnpLab.dnpdata method)@\spxentry{phase()}\spxextra{dnpLab.dnpdata method}}

\begin{fulllineitems}
\phantomsection\label{\detokenize{dnpData:dnpLab.dnpdata.phase}}\pysiglinewithargsret{\sphinxbfcode{\sphinxupquote{phase}}}{}{}
Return phase of dnpdata object
\begin{quote}\begin{description}
\item[{Returns}] \leavevmode
phase of data calculated from sum of imaginary divided by sum of real components

\item[{Return type}] \leavevmode
phase (float,int)

\end{description}\end{quote}

\end{fulllineitems}

\index{squeeze() (dnpLab.dnpdata method)@\spxentry{squeeze()}\spxextra{dnpLab.dnpdata method}}

\begin{fulllineitems}
\phantomsection\label{\detokenize{dnpData:dnpLab.dnpdata.squeeze}}\pysiglinewithargsret{\sphinxbfcode{\sphinxupquote{squeeze}}}{}{}
Remove all length 1 dimensions from data

\begin{sphinxadmonition}{warning}{Warning:}
Axes information is lost
\end{sphinxadmonition}

Example:
data.squeeze()

\end{fulllineitems}

\index{align() (dnpLab.dnpdata method)@\spxentry{align()}\spxextra{dnpLab.dnpdata method}}

\begin{fulllineitems}
\phantomsection\label{\detokenize{dnpData:dnpLab.dnpdata.align}}\pysiglinewithargsret{\sphinxbfcode{\sphinxupquote{align}}}{\emph{\DUrole{n}{b}}}{}
Align two data objects for numerical operations
\begin{quote}\begin{description}
\item[{Parameters}] \leavevmode
\sphinxstyleliteralstrong{\sphinxupquote{b}} \sphinxhyphen{}\sphinxhyphen{} Ojbect to align with self

\item[{Returns}] \leavevmode
self and b aligned data objects

\item[{Return type}] \leavevmode
tuple

\end{description}\end{quote}

\end{fulllineitems}

\index{argmax() (dnpLab.dnpdata method)@\spxentry{argmax()}\spxextra{dnpLab.dnpdata method}}

\begin{fulllineitems}
\phantomsection\label{\detokenize{dnpData:dnpLab.dnpdata.argmax}}\pysiglinewithargsret{\sphinxbfcode{\sphinxupquote{argmax}}}{\emph{\DUrole{n}{dim}}}{}
Return argmax for given dim

\end{fulllineitems}

\index{argmin() (dnpLab.dnpdata method)@\spxentry{argmin()}\spxextra{dnpLab.dnpdata method}}

\begin{fulllineitems}
\phantomsection\label{\detokenize{dnpData:dnpLab.dnpdata.argmin}}\pysiglinewithargsret{\sphinxbfcode{\sphinxupquote{argmin}}}{\emph{\DUrole{n}{dim}}}{}
Return argmin for given dim

\end{fulllineitems}

\index{chunk() (dnpLab.dnpdata method)@\spxentry{chunk()}\spxextra{dnpLab.dnpdata method}}

\begin{fulllineitems}
\phantomsection\label{\detokenize{dnpData:dnpLab.dnpdata.chunk}}\pysiglinewithargsret{\sphinxbfcode{\sphinxupquote{chunk}}}{\emph{\DUrole{n}{dim}}, \emph{\DUrole{n}{new\_dims}}, \emph{\DUrole{n}{new\_sizes}}}{}~
\begin{sphinxadmonition}{note}{Note:}
This is a placeholder for a function that\textquotesingle{}s not yet implemented
\end{sphinxadmonition}
\begin{quote}\begin{description}
\item[{Parameters}] \leavevmode\begin{itemize}
\item {} 
\sphinxstyleliteralstrong{\sphinxupquote{dim}} (\sphinxstyleliteralemphasis{\sphinxupquote{str}}) \sphinxhyphen{}\sphinxhyphen{} Assume that the dimension \sphinxtitleref{dim} is a direct product of the
dimensions given in \sphinxtitleref{new\_dims}, and chunk it out into those new
dimensions.

\item {} 
\sphinxstyleliteralstrong{\sphinxupquote{new\_dims}} (\sphinxstyleliteralemphasis{\sphinxupquote{list of str}}) \sphinxhyphen{}\sphinxhyphen{} 
The new dimensions to generate.  Note that one of the elements of
the list can be \sphinxtitleref{dim} if you like.

It\textquotesingle{}s assumed that the ordering of \sphinxtitleref{dim} is a direct product given
in C\sphinxhyphen{}ordering (\sphinxstyleemphasis{i.e.} the inner dimensions are listed last and the
outer dimensions are listed first \sphinxhyphen{}\sphinxhyphen{} here "inner" means that
changes to the index of the inner\sphinxhyphen{}most dimension correspond to
adjacent positions in memory and/or adjacent indeces in the
original dimension that you are chunking)


\item {} 
\sphinxstyleliteralstrong{\sphinxupquote{new\_sizes}} (\sphinxstyleliteralemphasis{\sphinxupquote{list of int}}) \sphinxhyphen{}\sphinxhyphen{} sizes of the new dimensions\textasciigrave{}

\end{itemize}

\item[{Returns}] \leavevmode
\sphinxstylestrong{self} \sphinxhyphen{}\sphinxhyphen{} The new nddata object.
Note that uniformly ascending or descending coordinates are manipulated in a rational way,
\sphinxstyleemphasis{e.g.} \sphinxtitleref{{[}1,2,3,4,5,6{]}} when chunked to a size of \sphinxtitleref{{[}2,3{]}} will yield
coordinates for the two new dimensions:
\sphinxtitleref{{[}1,4{]}} and \sphinxtitleref{{[}0,1,2{]}}.
Coordinates that are not uniformly ascending or descending will
yield and error and must be manually modified by the user.

\item[{Return type}] \leavevmode
nddata\_core

\end{description}\end{quote}

\end{fulllineitems}

\index{concatenate() (dnpLab.dnpdata method)@\spxentry{concatenate()}\spxextra{dnpLab.dnpdata method}}

\begin{fulllineitems}
\phantomsection\label{\detokenize{dnpData:dnpLab.dnpdata.concatenate}}\pysiglinewithargsret{\sphinxbfcode{\sphinxupquote{concatenate}}}{\emph{\DUrole{n}{b}}, \emph{\DUrole{n}{dim}}}{}
\end{fulllineitems}

\index{copy() (dnpLab.dnpdata method)@\spxentry{copy()}\spxextra{dnpLab.dnpdata method}}

\begin{fulllineitems}
\phantomsection\label{\detokenize{dnpData:dnpLab.dnpdata.copy}}\pysiglinewithargsret{\sphinxbfcode{\sphinxupquote{copy}}}{}{}
Return deepcopy of dnpdata object
\begin{quote}\begin{description}
\item[{Returns}] \leavevmode
deep copy of data object

\end{description}\end{quote}

\end{fulllineitems}

\index{get\_coord() (dnpLab.dnpdata method)@\spxentry{get\_coord()}\spxextra{dnpLab.dnpdata method}}

\begin{fulllineitems}
\phantomsection\label{\detokenize{dnpData:dnpLab.dnpdata.get_coord}}\pysiglinewithargsret{\sphinxbfcode{\sphinxupquote{get\_coord}}}{\emph{\DUrole{n}{dim}}}{}
Return coord corresponding to given dimension name
\begin{quote}\begin{description}
\item[{Parameters}] \leavevmode
\sphinxstyleliteralstrong{\sphinxupquote{dim}} (\sphinxstyleliteralemphasis{\sphinxupquote{str}}) \sphinxhyphen{}\sphinxhyphen{} Name of dim to retrieve coordinates from

\item[{Returns}] \leavevmode
array of coordinates

\item[{Return type}] \leavevmode
numpy.ndarray

\end{description}\end{quote}

\end{fulllineitems}

\index{index() (dnpLab.dnpdata method)@\spxentry{index()}\spxextra{dnpLab.dnpdata method}}

\begin{fulllineitems}
\phantomsection\label{\detokenize{dnpData:dnpLab.dnpdata.index}}\pysiglinewithargsret{\sphinxbfcode{\sphinxupquote{index}}}{\emph{\DUrole{n}{dim}}}{}
Find index of given dimension name

\end{fulllineitems}

\index{is\_sorted() (dnpLab.dnpdata method)@\spxentry{is\_sorted()}\spxextra{dnpLab.dnpdata method}}

\begin{fulllineitems}
\phantomsection\label{\detokenize{dnpData:dnpLab.dnpdata.is_sorted}}\pysiglinewithargsret{\sphinxbfcode{\sphinxupquote{is\_sorted}}}{\emph{\DUrole{n}{dim}}}{}
Determine if coords corresponding to give dim are sorted in ascending order
:param dim: Dimension to check if sorted
:type dim: str
\begin{quote}\begin{description}
\item[{Returns}] \leavevmode
True if sorted, False otherwise.

\item[{Return type}] \leavevmode
bool

\end{description}\end{quote}

\end{fulllineitems}

\index{merge\_attrs() (dnpLab.dnpdata method)@\spxentry{merge\_attrs()}\spxextra{dnpLab.dnpdata method}}

\begin{fulllineitems}
\phantomsection\label{\detokenize{dnpData:dnpLab.dnpdata.merge_attrs}}\pysiglinewithargsret{\sphinxbfcode{\sphinxupquote{merge\_attrs}}}{\emph{\DUrole{n}{b}}}{}
Merge the given dictionaries
\begin{quote}\begin{description}
\item[{Parameters}] \leavevmode
\sphinxstyleliteralstrong{\sphinxupquote{b}} (\sphinxstyleliteralemphasis{\sphinxupquote{nddata\_core}}) \sphinxhyphen{}\sphinxhyphen{} attributes to merge into object

\end{description}\end{quote}

\end{fulllineitems}

\index{new\_dim() (dnpLab.dnpdata method)@\spxentry{new\_dim()}\spxextra{dnpLab.dnpdata method}}

\begin{fulllineitems}
\phantomsection\label{\detokenize{dnpData:dnpLab.dnpdata.new_dim}}\pysiglinewithargsret{\sphinxbfcode{\sphinxupquote{new\_dim}}}{\emph{\DUrole{n}{dim}}, \emph{\DUrole{n}{coord}}}{}
\end{fulllineitems}

\index{rename() (dnpLab.dnpdata method)@\spxentry{rename()}\spxextra{dnpLab.dnpdata method}}

\begin{fulllineitems}
\phantomsection\label{\detokenize{dnpData:dnpLab.dnpdata.rename}}\pysiglinewithargsret{\sphinxbfcode{\sphinxupquote{rename}}}{\emph{\DUrole{n}{dim}}, \emph{\DUrole{n}{new\_name}}}{}
Rename dim
\begin{quote}\begin{description}
\item[{Parameters}] \leavevmode\begin{itemize}
\item {} 
\sphinxstyleliteralstrong{\sphinxupquote{dim}} (\sphinxstyleliteralemphasis{\sphinxupquote{str}}) \sphinxhyphen{}\sphinxhyphen{} Name of dimension to rename

\item {} 
\sphinxstyleliteralstrong{\sphinxupquote{new\_name}} (\sphinxstyleliteralemphasis{\sphinxupquote{str}}) \sphinxhyphen{}\sphinxhyphen{} New name for dim

\end{itemize}

\end{description}\end{quote}

\end{fulllineitems}

\index{reorder() (dnpLab.dnpdata method)@\spxentry{reorder()}\spxextra{dnpLab.dnpdata method}}

\begin{fulllineitems}
\phantomsection\label{\detokenize{dnpData:dnpLab.dnpdata.reorder}}\pysiglinewithargsret{\sphinxbfcode{\sphinxupquote{reorder}}}{\emph{\DUrole{n}{dims}}}{}
\end{fulllineitems}

\index{size() (dnpLab.dnpdata property)@\spxentry{size()}\spxextra{dnpLab.dnpdata property}}

\begin{fulllineitems}
\phantomsection\label{\detokenize{dnpData:dnpLab.dnpdata.size}}\pysigline{\sphinxbfcode{\sphinxupquote{property }}\sphinxbfcode{\sphinxupquote{size}}}
Returns values.size. Total number of elements in numpy array.

\end{fulllineitems}

\index{smoosh() (dnpLab.dnpdata method)@\spxentry{smoosh()}\spxextra{dnpLab.dnpdata method}}

\begin{fulllineitems}
\phantomsection\label{\detokenize{dnpData:dnpLab.dnpdata.smoosh}}\pysiglinewithargsret{\sphinxbfcode{\sphinxupquote{smoosh}}}{\emph{\DUrole{n}{old\_dims}}, \emph{\DUrole{n}{new\_name}}}{}~
\begin{sphinxadmonition}{note}{Note:}
Not yet implemented.
\end{sphinxadmonition}

\sphinxtitleref{smoosh} does the opposite of \sphinxtitleref{chunk} \sphinxhyphen{}\sphinxhyphen{} see :func\textasciigrave{}:\textasciitilde{}nddata\_core.chunk\textasciigrave{}

\end{fulllineitems}

\index{sort() (dnpLab.dnpdata method)@\spxentry{sort()}\spxextra{dnpLab.dnpdata method}}

\begin{fulllineitems}
\phantomsection\label{\detokenize{dnpData:dnpLab.dnpdata.sort}}\pysiglinewithargsret{\sphinxbfcode{\sphinxupquote{sort}}}{\emph{\DUrole{n}{dim}}}{}
Sort the coords corresponding to the given dim in ascending order
\begin{quote}\begin{description}
\item[{Parameters}] \leavevmode
\sphinxstyleliteralstrong{\sphinxupquote{dim}} (\sphinxstyleliteralemphasis{\sphinxupquote{str}}) \sphinxhyphen{}\sphinxhyphen{} dimension to sort

\end{description}\end{quote}

\end{fulllineitems}

\index{sort\_dims() (dnpLab.dnpdata method)@\spxentry{sort\_dims()}\spxextra{dnpLab.dnpdata method}}

\begin{fulllineitems}
\phantomsection\label{\detokenize{dnpData:dnpLab.dnpdata.sort_dims}}\pysiglinewithargsret{\sphinxbfcode{\sphinxupquote{sort\_dims}}}{}{}
Sort the dimensions

\end{fulllineitems}

\index{sum() (dnpLab.dnpdata method)@\spxentry{sum()}\spxextra{dnpLab.dnpdata method}}

\begin{fulllineitems}
\phantomsection\label{\detokenize{dnpData:dnpLab.dnpdata.sum}}\pysiglinewithargsret{\sphinxbfcode{\sphinxupquote{sum}}}{\emph{\DUrole{n}{dim}}}{}
Perform sum down given dimension

\end{fulllineitems}


\end{fulllineitems}



\subsection{dnpdata\_collection (Workspace)}
\label{\detokenize{dnpData:dnpdata-collection-workspace}}
To store multiple data objects, the user can create a workspace which is a dict like object for storing dnpdata objects.

\begin{sphinxVerbatim}[commandchars=\\\{\}]
\PYG{k+kn}{import} \PYG{n+nn}{dnpLab} \PYG{k}{as} \PYG{n+nn}{dnp}

\PYG{n}{ws} \PYG{o}{=} \PYG{n}{dnp}\PYG{o}{.}\PYG{n}{create\PYGZus{}workspace}\PYG{p}{(}\PYG{p}{)}
\PYG{n}{ws}\PYG{p}{[}\PYG{l+s+s1}{\PYGZsq{}}\PYG{l+s+s1}{raw}\PYG{l+s+s1}{\PYGZsq{}}\PYG{p}{]} \PYG{o}{=} \PYG{n}{data}
\end{sphinxVerbatim}


\subsection{Processing Buffer}
\label{\detokenize{dnpData:processing-buffer}}
The workspace has an attribute called processing\_buffer. The processing buffer indicates for functions which operate on the workspace, which dnpdata object should be operated on. By default, the processing\_buffer is called "proc".

\begin{sphinxVerbatim}[commandchars=\\\{\}]
\PYG{n}{ws}\PYG{o}{.}\PYG{n}{copy}\PYG{p}{(}\PYG{l+s+s1}{\PYGZsq{}}\PYG{l+s+s1}{raw}\PYG{l+s+s1}{\PYGZsq{}}\PYG{p}{,} \PYG{l+s+s1}{\PYGZsq{}}\PYG{l+s+s1}{proc}\PYG{l+s+s1}{\PYGZsq{}}\PYG{p}{)} \PYG{c+c1}{\PYGZsh{} copy some data into the default processing buffer}
\end{sphinxVerbatim}

At any time, the processing buffer can be changed, however you need to make sure to move data into the processing buffer before any processing steps.

\begin{sphinxVerbatim}[commandchars=\\\{\}]
\PYG{n}{ws}\PYG{o}{.}\PYG{n}{processing\PYGZus{}buffer} \PYG{o}{=} \PYG{l+s+s1}{\PYGZsq{}}\PYG{l+s+s1}{new\PYGZus{}proc}\PYG{l+s+s1}{\PYGZsq{}}
\end{sphinxVerbatim}


\subsection{Saving the Workspace}
\label{\detokenize{dnpData:saving-the-workspace}}
The workspace can be saved in h5 format with the saveh5 function:

\begin{sphinxVerbatim}[commandchars=\\\{\}]
\PYG{n}{dnplab}\PYG{o}{.}\PYG{n}{dnpImport}\PYG{o}{.}\PYG{n}{saveh5}\PYG{p}{(}\PYG{l+s+s1}{\PYGZsq{}}\PYG{l+s+s1}{test.h5}\PYG{l+s+s1}{\PYGZsq{}}\PYG{p}{,} \PYG{n}{ws}\PYG{p}{)}
\end{sphinxVerbatim}


\subsection{Loading the Workspace}
\label{\detokenize{dnpData:loading-the-workspace}}
A workspace can also be loaded with the loadh5 function.

\begin{sphinxVerbatim}[commandchars=\\\{\}]
\PYG{n}{dnplab}\PYG{o}{.}\PYG{n}{dnpImport}\PYG{o}{.}\PYG{n}{loadh5}\PYG{p}{(}\PYG{l+s+s1}{\PYGZsq{}}\PYG{l+s+s1}{test.h5}\PYG{l+s+s1}{\PYGZsq{}}\PYG{p}{)}
\end{sphinxVerbatim}


\subsection{dnpdata\_collection Methods}
\label{\detokenize{dnpData:dnpdata-collection-methods}}\index{dnpdata\_collection (class in dnpLab)@\spxentry{dnpdata\_collection}\spxextra{class in dnpLab}}

\begin{fulllineitems}
\phantomsection\label{\detokenize{dnpData:dnpLab.dnpdata_collection}}\pysiglinewithargsret{\sphinxbfcode{\sphinxupquote{class }}\sphinxcode{\sphinxupquote{dnpLab.}}\sphinxbfcode{\sphinxupquote{dnpdata\_collection}}}{\emph{\DUrole{o}{*}\DUrole{n}{args}}, \emph{\DUrole{o}{**}\DUrole{n}{kwargs}}}{}
Bases: \sphinxcode{\sphinxupquote{collections.abc.MutableMapping}}

Dictionary\sphinxhyphen{}like workspace object for storing dnpdata objects
\index{copy() (dnpLab.dnpdata\_collection method)@\spxentry{copy()}\spxextra{dnpLab.dnpdata\_collection method}}

\begin{fulllineitems}
\phantomsection\label{\detokenize{dnpData:dnpLab.dnpdata_collection.copy}}\pysiglinewithargsret{\sphinxbfcode{\sphinxupquote{copy}}}{\emph{\DUrole{n}{key}}, \emph{\DUrole{n}{new\_key}\DUrole{o}{=}\DUrole{default_value}{None}}}{}
Copy data from key to new\_key. If new\_key is not given, by default key will be copied to processing buffer
\begin{quote}\begin{description}
\item[{Parameters}] \leavevmode\begin{itemize}
\item {} 
\sphinxstyleliteralstrong{\sphinxupquote{key}} (\sphinxstyleliteralemphasis{\sphinxupquote{str}}) \sphinxhyphen{}\sphinxhyphen{} Key to be copied

\item {} 
\sphinxstyleliteralstrong{\sphinxupquote{new\_key}} (\sphinxstyleliteralemphasis{\sphinxupquote{str}}\sphinxstyleliteralemphasis{\sphinxupquote{, }}\sphinxstyleliteralemphasis{\sphinxupquote{None}}) \sphinxhyphen{}\sphinxhyphen{} New key for copied data

\end{itemize}

\end{description}\end{quote}

\end{fulllineitems}

\index{move() (dnpLab.dnpdata\_collection method)@\spxentry{move()}\spxextra{dnpLab.dnpdata\_collection method}}

\begin{fulllineitems}
\phantomsection\label{\detokenize{dnpData:dnpLab.dnpdata_collection.move}}\pysiglinewithargsret{\sphinxbfcode{\sphinxupquote{move}}}{\emph{\DUrole{n}{key}}, \emph{\DUrole{n}{new\_key}}}{}
Move data from key to new\_key
\begin{quote}\begin{description}
\item[{Parameters}] \leavevmode\begin{itemize}
\item {} 
\sphinxstyleliteralstrong{\sphinxupquote{key}} (\sphinxstyleliteralemphasis{\sphinxupquote{str}}) \sphinxhyphen{}\sphinxhyphen{} Name of data to move

\item {} 
\sphinxstyleliteralstrong{\sphinxupquote{new\_key}} (\sphinxstyleliteralemphasis{\sphinxupquote{str}}) \sphinxhyphen{}\sphinxhyphen{} Name of new key to move data

\end{itemize}

\end{description}\end{quote}

\end{fulllineitems}

\index{pop() (dnpLab.dnpdata\_collection method)@\spxentry{pop()}\spxextra{dnpLab.dnpdata\_collection method}}

\begin{fulllineitems}
\phantomsection\label{\detokenize{dnpData:dnpLab.dnpdata_collection.pop}}\pysiglinewithargsret{\sphinxbfcode{\sphinxupquote{pop}}}{\emph{\DUrole{n}{key}}}{}
Pop key. Removes data corresponding to key.

\end{fulllineitems}

\index{dict() (dnpLab.dnpdata\_collection method)@\spxentry{dict()}\spxextra{dnpLab.dnpdata\_collection method}}

\begin{fulllineitems}
\phantomsection\label{\detokenize{dnpData:dnpLab.dnpdata_collection.dict}}\pysiglinewithargsret{\sphinxbfcode{\sphinxupquote{dict}}}{}{}
Return dictionary for storing data in dnpdata\_collection

\end{fulllineitems}

\index{clear() (dnpLab.dnpdata\_collection method)@\spxentry{clear()}\spxextra{dnpLab.dnpdata\_collection method}}

\begin{fulllineitems}
\phantomsection\label{\detokenize{dnpData:dnpLab.dnpdata_collection.clear}}\pysiglinewithargsret{\sphinxbfcode{\sphinxupquote{clear}}}{}{}
Removes all items

\end{fulllineitems}

\index{items() (dnpLab.dnpdata\_collection method)@\spxentry{items()}\spxextra{dnpLab.dnpdata\_collection method}}

\begin{fulllineitems}
\phantomsection\label{\detokenize{dnpData:dnpLab.dnpdata_collection.items}}\pysiglinewithargsret{\sphinxbfcode{\sphinxupquote{items}}}{}{}
Return items

\end{fulllineitems}

\index{keys() (dnpLab.dnpdata\_collection method)@\spxentry{keys()}\spxextra{dnpLab.dnpdata\_collection method}}

\begin{fulllineitems}
\phantomsection\label{\detokenize{dnpData:dnpLab.dnpdata_collection.keys}}\pysiglinewithargsret{\sphinxbfcode{\sphinxupquote{keys}}}{}{}
Return keys.

\end{fulllineitems}

\index{popitem() (dnpLab.dnpdata\_collection method)@\spxentry{popitem()}\spxextra{dnpLab.dnpdata\_collection method}}

\begin{fulllineitems}
\phantomsection\label{\detokenize{dnpData:dnpLab.dnpdata_collection.popitem}}\pysiglinewithargsret{\sphinxbfcode{\sphinxupquote{popitem}}}{}{}
Pops item from end of dnpdata\_collection
\begin{quote}\begin{description}
\item[{Returns}] \leavevmode
key, item pair that was removed

\item[{Return type}] \leavevmode
tuple

\end{description}\end{quote}

\end{fulllineitems}

\index{values() (dnpLab.dnpdata\_collection method)@\spxentry{values()}\spxextra{dnpLab.dnpdata\_collection method}}

\begin{fulllineitems}
\phantomsection\label{\detokenize{dnpData:dnpLab.dnpdata_collection.values}}\pysiglinewithargsret{\sphinxbfcode{\sphinxupquote{values}}}{}{}
Return Values

\end{fulllineitems}

\index{add() (dnpLab.dnpdata\_collection method)@\spxentry{add()}\spxextra{dnpLab.dnpdata\_collection method}}

\begin{fulllineitems}
\phantomsection\label{\detokenize{dnpData:dnpLab.dnpdata_collection.add}}\pysiglinewithargsret{\sphinxbfcode{\sphinxupquote{add}}}{\emph{\DUrole{n}{key}}, \emph{\DUrole{n}{data}}}{}
Adds new data
\begin{quote}\begin{description}
\item[{Parameters}] \leavevmode\begin{itemize}
\item {} 
\sphinxstyleliteralstrong{\sphinxupquote{key}} (\sphinxstyleliteralemphasis{\sphinxupquote{str}}) \sphinxhyphen{}\sphinxhyphen{} key corresponding to new data

\item {} 
\sphinxstyleliteralstrong{\sphinxupquote{data}} ({\hyperref[\detokenize{dnpData:dnpLab.dnpdata}]{\sphinxcrossref{\sphinxstyleliteralemphasis{\sphinxupquote{dnpdata}}}}}) \sphinxhyphen{}\sphinxhyphen{} data object corresponding to key

\end{itemize}

\end{description}\end{quote}

\end{fulllineitems}


\end{fulllineitems}



\section{dnpImport}
\label{\detokenize{dnpImport:dnpimport}}\label{\detokenize{dnpImport:import}}\label{\detokenize{dnpImport::doc}}

\subsection{Topspin Module}
\label{\detokenize{dnpImport:module-dnpLab.dnpImport.topspin}}\label{\detokenize{dnpImport:topspin-module}}\label{\detokenize{dnpImport:topspin}}\index{module@\spxentry{module}!dnpLab.dnpImport.topspin@\spxentry{dnpLab.dnpImport.topspin}}\index{dnpLab.dnpImport.topspin@\spxentry{dnpLab.dnpImport.topspin}!module@\spxentry{module}}\index{find\_group\_delay() (in module dnpLab.dnpImport.topspin)@\spxentry{find\_group\_delay()}\spxextra{in module dnpLab.dnpImport.topspin}}

\begin{fulllineitems}
\phantomsection\label{\detokenize{dnpImport:dnpLab.dnpImport.topspin.find_group_delay}}\pysiglinewithargsret{\sphinxcode{\sphinxupquote{dnpLab.dnpImport.topspin.}}\sphinxbfcode{\sphinxupquote{find\_group\_delay}}}{\emph{\DUrole{n}{decim}}, \emph{\DUrole{n}{dspfvs}}}{}
Determine group delay from tables
\begin{quote}\begin{description}
\item[{Parameters}] \leavevmode\begin{itemize}
\item {} 
\sphinxstyleliteralstrong{\sphinxupquote{decim}} \sphinxhyphen{}\sphinxhyphen{} Decimation factor of the digital filter (factor by which oversampling rate exeeds sampling rate).

\item {} 
\sphinxstyleliteralstrong{\sphinxupquote{dspfvs}} \sphinxhyphen{}\sphinxhyphen{} Firmware version for Bruker Console.

\end{itemize}

\item[{Returns}] \leavevmode
Group delay. Number of points FID is shifted by DSP. The ceiling of this number (group delay rounded up) is the number of points should be removed from the start of the FID.

\item[{Return type}] \leavevmode
float

\end{description}\end{quote}

\end{fulllineitems}

\index{load\_title() (in module dnpLab.dnpImport.topspin)@\spxentry{load\_title()}\spxextra{in module dnpLab.dnpImport.topspin}}

\begin{fulllineitems}
\phantomsection\label{\detokenize{dnpImport:dnpLab.dnpImport.topspin.load_title}}\pysiglinewithargsret{\sphinxcode{\sphinxupquote{dnpLab.dnpImport.topspin.}}\sphinxbfcode{\sphinxupquote{load\_title}}}{\emph{\DUrole{n}{path}}, \emph{\DUrole{n}{expNum}\DUrole{o}{=}\DUrole{default_value}{1}}, \emph{\DUrole{n}{titlePath}\DUrole{o}{=}\DUrole{default_value}{\textquotesingle{}pdata/1\textquotesingle{}}}, \emph{\DUrole{n}{titleFilename}\DUrole{o}{=}\DUrole{default_value}{\textquotesingle{}title\textquotesingle{}}}}{}
Import Topspin Experiment Title File
\begin{quote}\begin{description}
\item[{Parameters}] \leavevmode\begin{itemize}
\item {} 
\sphinxstyleliteralstrong{\sphinxupquote{path}} (\sphinxstyleliteralemphasis{\sphinxupquote{str}}) \sphinxhyphen{}\sphinxhyphen{} Directory of title

\item {} 
\sphinxstyleliteralstrong{\sphinxupquote{expNum}} (\sphinxstyleliteralemphasis{\sphinxupquote{int}}) \sphinxhyphen{}\sphinxhyphen{} Experiment number to return title

\item {} 
\sphinxstyleliteralstrong{\sphinxupquote{titlePath}} (\sphinxstyleliteralemphasis{\sphinxupquote{str}}) \sphinxhyphen{}\sphinxhyphen{} Path within experiment of title

\item {} 
\sphinxstyleliteralstrong{\sphinxupquote{titleFilename}} (\sphinxstyleliteralemphasis{\sphinxupquote{str}}) \sphinxhyphen{}\sphinxhyphen{} filename of title

\end{itemize}

\item[{Returns}] \leavevmode
Contents of experiment title file

\item[{Return type}] \leavevmode
str

\end{description}\end{quote}

\end{fulllineitems}

\index{load\_acqu() (in module dnpLab.dnpImport.topspin)@\spxentry{load\_acqu()}\spxextra{in module dnpLab.dnpImport.topspin}}

\begin{fulllineitems}
\phantomsection\label{\detokenize{dnpImport:dnpLab.dnpImport.topspin.load_acqu}}\pysiglinewithargsret{\sphinxcode{\sphinxupquote{dnpLab.dnpImport.topspin.}}\sphinxbfcode{\sphinxupquote{load\_acqu}}}{\emph{\DUrole{n}{path}}, \emph{\DUrole{n}{expNum}\DUrole{o}{=}\DUrole{default_value}{1}}, \emph{\DUrole{n}{paramFilename}\DUrole{o}{=}\DUrole{default_value}{\textquotesingle{}acqus\textquotesingle{}}}}{}
Import Topspin JCAMPDX file
\begin{quote}\begin{description}
\item[{Parameters}] \leavevmode\begin{itemize}
\item {} 
\sphinxstyleliteralstrong{\sphinxupquote{path}} (\sphinxstyleliteralemphasis{\sphinxupquote{str}}) \sphinxhyphen{}\sphinxhyphen{} directory of acqusition file

\item {} 
\sphinxstyleliteralstrong{\sphinxupquote{expNum}} (\sphinxstyleliteralemphasis{\sphinxupquote{int}}) \sphinxhyphen{}\sphinxhyphen{} Experiment number

\item {} 
\sphinxstyleliteralstrong{\sphinxupquote{paramFilename}} (\sphinxstyleliteralemphasis{\sphinxupquote{str}}) \sphinxhyphen{}\sphinxhyphen{} Acqusition parameters filename

\end{itemize}

\item[{Returns}] \leavevmode
Dictionary of acqusition parameters

\item[{Return type}] \leavevmode
dict

\end{description}\end{quote}

\end{fulllineitems}

\index{load\_proc() (in module dnpLab.dnpImport.topspin)@\spxentry{load\_proc()}\spxextra{in module dnpLab.dnpImport.topspin}}

\begin{fulllineitems}
\phantomsection\label{\detokenize{dnpImport:dnpLab.dnpImport.topspin.load_proc}}\pysiglinewithargsret{\sphinxcode{\sphinxupquote{dnpLab.dnpImport.topspin.}}\sphinxbfcode{\sphinxupquote{load\_proc}}}{\emph{\DUrole{n}{path}}, \emph{\DUrole{n}{expNum}\DUrole{o}{=}\DUrole{default_value}{1}}, \emph{\DUrole{n}{procNum}\DUrole{o}{=}\DUrole{default_value}{1}}, \emph{\DUrole{n}{paramFilename}\DUrole{o}{=}\DUrole{default_value}{\textquotesingle{}procs\textquotesingle{}}}}{}
\end{fulllineitems}

\index{dir\_data\_type() (in module dnpLab.dnpImport.topspin)@\spxentry{dir\_data\_type()}\spxextra{in module dnpLab.dnpImport.topspin}}

\begin{fulllineitems}
\phantomsection\label{\detokenize{dnpImport:dnpLab.dnpImport.topspin.dir_data_type}}\pysiglinewithargsret{\sphinxcode{\sphinxupquote{dnpLab.dnpImport.topspin.}}\sphinxbfcode{\sphinxupquote{dir\_data\_type}}}{\emph{\DUrole{n}{path}}, \emph{\DUrole{n}{expNum}}}{}
Determine type of data in directory
\begin{quote}\begin{description}
\item[{Parameters}] \leavevmode\begin{itemize}
\item {} 
\sphinxstyleliteralstrong{\sphinxupquote{path}} (\sphinxstyleliteralemphasis{\sphinxupquote{str}}) \sphinxhyphen{}\sphinxhyphen{} Directory of data

\item {} 
\sphinxstyleliteralstrong{\sphinxupquote{expNum}} (\sphinxstyleliteralemphasis{\sphinxupquote{int}}) \sphinxhyphen{}\sphinxhyphen{} Experiment number

\end{itemize}

\item[{Returns}] \leavevmode
String identifying filetype

\item[{Return type}] \leavevmode
str

\end{description}\end{quote}

\end{fulllineitems}

\index{import\_topspin() (in module dnpLab.dnpImport.topspin)@\spxentry{import\_topspin()}\spxextra{in module dnpLab.dnpImport.topspin}}

\begin{fulllineitems}
\phantomsection\label{\detokenize{dnpImport:dnpLab.dnpImport.topspin.import_topspin}}\pysiglinewithargsret{\sphinxcode{\sphinxupquote{dnpLab.dnpImport.topspin.}}\sphinxbfcode{\sphinxupquote{import\_topspin}}}{\emph{\DUrole{n}{path}}, \emph{\DUrole{n}{expNum}}, \emph{\DUrole{n}{paramFilename}\DUrole{o}{=}\DUrole{default_value}{\textquotesingle{}acqus\textquotesingle{}}}}{}
Import topspin data and return dnpdata object
\begin{quote}\begin{description}
\item[{Parameters}] \leavevmode\begin{itemize}
\item {} 
\sphinxstyleliteralstrong{\sphinxupquote{path}} (\sphinxstyleliteralemphasis{\sphinxupquote{str}}) \sphinxhyphen{}\sphinxhyphen{} Directory of data

\item {} 
\sphinxstyleliteralstrong{\sphinxupquote{expNum}} (\sphinxstyleliteralemphasis{\sphinxupquote{int}}) \sphinxhyphen{}\sphinxhyphen{} Experiment number

\item {} 
\sphinxstyleliteralstrong{\sphinxupquote{paramFilename}} (\sphinxstyleliteralemphasis{\sphinxupquote{str}}) \sphinxhyphen{}\sphinxhyphen{} Parameters filename

\end{itemize}

\item[{Returns}] \leavevmode
topspin data

\item[{Return type}] \leavevmode
{\hyperref[\detokenize{dnpData:dnpLab.dnpdata}]{\sphinxcrossref{dnpdata}}}

\end{description}\end{quote}

\end{fulllineitems}

\index{topspin\_fid() (in module dnpLab.dnpImport.topspin)@\spxentry{topspin\_fid()}\spxextra{in module dnpLab.dnpImport.topspin}}

\begin{fulllineitems}
\phantomsection\label{\detokenize{dnpImport:dnpLab.dnpImport.topspin.topspin_fid}}\pysiglinewithargsret{\sphinxcode{\sphinxupquote{dnpLab.dnpImport.topspin.}}\sphinxbfcode{\sphinxupquote{topspin\_fid}}}{\emph{\DUrole{n}{path}}, \emph{\DUrole{n}{expNum}}, \emph{\DUrole{n}{paramFilename}\DUrole{o}{=}\DUrole{default_value}{\textquotesingle{}acqus\textquotesingle{}}}}{}
Import topspin fid data and return dnpdata object
\begin{quote}\begin{description}
\item[{Parameters}] \leavevmode\begin{itemize}
\item {} 
\sphinxstyleliteralstrong{\sphinxupquote{path}} (\sphinxstyleliteralemphasis{\sphinxupquote{str}}) \sphinxhyphen{}\sphinxhyphen{} Directory of data

\item {} 
\sphinxstyleliteralstrong{\sphinxupquote{expNum}} (\sphinxstyleliteralemphasis{\sphinxupquote{int}}) \sphinxhyphen{}\sphinxhyphen{} Experiment number

\item {} 
\sphinxstyleliteralstrong{\sphinxupquote{paramFilename}} (\sphinxstyleliteralemphasis{\sphinxupquote{str}}) \sphinxhyphen{}\sphinxhyphen{} Parameters filename

\end{itemize}

\item[{Returns}] \leavevmode
Topspin data

\item[{Return type}] \leavevmode
{\hyperref[\detokenize{dnpData:dnpLab.dnpdata}]{\sphinxcrossref{dnpdata}}}

\end{description}\end{quote}

\end{fulllineitems}

\index{topspin\_jcamp\_dx() (in module dnpLab.dnpImport.topspin)@\spxentry{topspin\_jcamp\_dx()}\spxextra{in module dnpLab.dnpImport.topspin}}

\begin{fulllineitems}
\phantomsection\label{\detokenize{dnpImport:dnpLab.dnpImport.topspin.topspin_jcamp_dx}}\pysiglinewithargsret{\sphinxcode{\sphinxupquote{dnpLab.dnpImport.topspin.}}\sphinxbfcode{\sphinxupquote{topspin\_jcamp\_dx}}}{\emph{\DUrole{n}{path}}}{}
Return the contents of topspin JCAMP\sphinxhyphen{}DX file as dictionary
\begin{quote}\begin{description}
\item[{Parameters}] \leavevmode
\sphinxstyleliteralstrong{\sphinxupquote{path}} \sphinxhyphen{}\sphinxhyphen{} Path to file

\item[{Returns}] \leavevmode
Dictionary of JCAMP\sphinxhyphen{}DX file

\item[{Return type}] \leavevmode
dict

\end{description}\end{quote}

\end{fulllineitems}

\index{topspin\_vdlist() (in module dnpLab.dnpImport.topspin)@\spxentry{topspin\_vdlist()}\spxextra{in module dnpLab.dnpImport.topspin}}

\begin{fulllineitems}
\phantomsection\label{\detokenize{dnpImport:dnpLab.dnpImport.topspin.topspin_vdlist}}\pysiglinewithargsret{\sphinxcode{\sphinxupquote{dnpLab.dnpImport.topspin.}}\sphinxbfcode{\sphinxupquote{topspin\_vdlist}}}{\emph{\DUrole{n}{path}}, \emph{\DUrole{n}{expNum}}}{}
Return topspin vdlist
\begin{quote}\begin{description}
\item[{Parameters}] \leavevmode\begin{itemize}
\item {} 
\sphinxstyleliteralstrong{\sphinxupquote{Path}} (\sphinxstyleliteralemphasis{\sphinxupquote{str}}) \sphinxhyphen{}\sphinxhyphen{} Directory of data

\item {} 
\sphinxstyleliteralstrong{\sphinxupquote{expNum}} (\sphinxstyleliteralemphasis{\sphinxupquote{int}}) \sphinxhyphen{}\sphinxhyphen{} Experiment number

\end{itemize}

\item[{Returns}] \leavevmode
vdlist as numpy array

\item[{Return type}] \leavevmode
numpy.ndarray

\end{description}\end{quote}

\end{fulllineitems}

\index{import\_ser() (in module dnpLab.dnpImport.topspin)@\spxentry{import\_ser()}\spxextra{in module dnpLab.dnpImport.topspin}}

\begin{fulllineitems}
\phantomsection\label{\detokenize{dnpImport:dnpLab.dnpImport.topspin.import_ser}}\pysiglinewithargsret{\sphinxcode{\sphinxupquote{dnpLab.dnpImport.topspin.}}\sphinxbfcode{\sphinxupquote{import\_ser}}}{\emph{\DUrole{n}{path}}, \emph{\DUrole{n}{expNum}}, \emph{\DUrole{n}{paramFilename}\DUrole{o}{=}\DUrole{default_value}{\textquotesingle{}acqus\textquotesingle{}}}}{}
Import topspin ser file
\begin{quote}\begin{description}
\item[{Parameters}] \leavevmode\begin{itemize}
\item {} 
\sphinxstyleliteralstrong{\sphinxupquote{path}} (\sphinxstyleliteralemphasis{\sphinxupquote{str}}) \sphinxhyphen{}\sphinxhyphen{} Directory of data

\item {} 
\sphinxstyleliteralstrong{\sphinxupquote{expNum}} (\sphinxstyleliteralemphasis{\sphinxupquote{int}}) \sphinxhyphen{}\sphinxhyphen{} Experiment number

\item {} 
\sphinxstyleliteralstrong{\sphinxupquote{paramFilename}} (\sphinxstyleliteralemphasis{\sphinxupquote{str}}) \sphinxhyphen{}\sphinxhyphen{} Filename of parameters file

\end{itemize}

\item[{Returns}] \leavevmode
Topspin data

\item[{Return type}] \leavevmode
{\hyperref[\detokenize{dnpData:dnpLab.dnpdata}]{\sphinxcrossref{dnpdata}}}

\end{description}\end{quote}

\end{fulllineitems}

\index{topspin\_ser\_phase\_cycle() (in module dnpLab.dnpImport.topspin)@\spxentry{topspin\_ser\_phase\_cycle()}\spxextra{in module dnpLab.dnpImport.topspin}}

\begin{fulllineitems}
\phantomsection\label{\detokenize{dnpImport:dnpLab.dnpImport.topspin.topspin_ser_phase_cycle}}\pysiglinewithargsret{\sphinxcode{\sphinxupquote{dnpLab.dnpImport.topspin.}}\sphinxbfcode{\sphinxupquote{topspin\_ser\_phase\_cycle}}}{\emph{\DUrole{n}{path}}, \emph{\DUrole{n}{expNum}}, \emph{\DUrole{n}{paramFilename}\DUrole{o}{=}\DUrole{default_value}{\textquotesingle{}acqus\textquotesingle{}}}}{}
Import Topspin data with phase cycle saved as different dimension
\begin{quote}\begin{description}
\item[{Parameters}] \leavevmode\begin{itemize}
\item {} 
\sphinxstyleliteralstrong{\sphinxupquote{path}} (\sphinxstyleliteralemphasis{\sphinxupquote{str}}) \sphinxhyphen{}\sphinxhyphen{} Directory of data

\item {} 
\sphinxstyleliteralstrong{\sphinxupquote{expNum}} (\sphinxstyleliteralemphasis{\sphinxupquote{int}}) \sphinxhyphen{}\sphinxhyphen{} Experiment number

\item {} 
\sphinxstyleliteralstrong{\sphinxupquote{paramFilename}} (\sphinxstyleliteralemphasis{\sphinxupquote{str}}) \sphinxhyphen{}\sphinxhyphen{} Filename of parameters file

\end{itemize}

\item[{Returns}] \leavevmode
Topspin data

\item[{Return type}] \leavevmode
{\hyperref[\detokenize{dnpData:dnpLab.dnpdata}]{\sphinxcrossref{dnpdata}}}

\end{description}\end{quote}

\end{fulllineitems}

\index{import\_topspin\_dir() (in module dnpLab.dnpImport.topspin)@\spxentry{import\_topspin\_dir()}\spxextra{in module dnpLab.dnpImport.topspin}}

\begin{fulllineitems}
\phantomsection\label{\detokenize{dnpImport:dnpLab.dnpImport.topspin.import_topspin_dir}}\pysiglinewithargsret{\sphinxcode{\sphinxupquote{dnpLab.dnpImport.topspin.}}\sphinxbfcode{\sphinxupquote{import\_topspin\_dir}}}{\emph{\DUrole{n}{path}}}{}
Import directory of Topspin data and return as dictionary
\begin{quote}\begin{description}
\item[{Parameters}] \leavevmode
\sphinxstyleliteralstrong{\sphinxupquote{path}} (\sphinxstyleliteralemphasis{\sphinxupquote{str}}) \sphinxhyphen{}\sphinxhyphen{} Directory of data

\item[{Returns}] \leavevmode
Topspin data. Keys correspond to folder name. Values correspond to dnpdata with topspin data for each folder.

\item[{Return type}] \leavevmode
dict

\end{description}\end{quote}

\end{fulllineitems}



\subsection{(Open) VnmrJ Module}
\label{\detokenize{dnpImport:module-dnpLab.dnpImport.vnmrj}}\label{\detokenize{dnpImport:open-vnmrj-module}}\label{\detokenize{dnpImport:vnmrj}}\index{module@\spxentry{module}!dnpLab.dnpImport.vnmrj@\spxentry{dnpLab.dnpImport.vnmrj}}\index{dnpLab.dnpImport.vnmrj@\spxentry{dnpLab.dnpImport.vnmrj}!module@\spxentry{module}}\index{array\_coords() (in module dnpLab.dnpImport.vnmrj)@\spxentry{array\_coords()}\spxextra{in module dnpLab.dnpImport.vnmrj}}

\begin{fulllineitems}
\phantomsection\label{\detokenize{dnpImport:dnpLab.dnpImport.vnmrj.array_coords}}\pysiglinewithargsret{\sphinxcode{\sphinxupquote{dnpLab.dnpImport.vnmrj.}}\sphinxbfcode{\sphinxupquote{array\_coords}}}{\emph{\DUrole{n}{attrs}}}{}
Return array dimension coords from parameters dictionary
\begin{quote}\begin{description}
\item[{Parameters}] \leavevmode
\sphinxstyleliteralstrong{\sphinxupquote{attrs}} (\sphinxstyleliteralemphasis{\sphinxupquote{dict}}) \sphinxhyphen{}\sphinxhyphen{} Dictionary of procpar parameters

\item[{Returns}] \leavevmode
dim and coord for array

\item[{Return type}] \leavevmode
tuple

\end{description}\end{quote}

\end{fulllineitems}

\index{import\_fid() (in module dnpLab.dnpImport.vnmrj)@\spxentry{import\_fid()}\spxextra{in module dnpLab.dnpImport.vnmrj}}

\begin{fulllineitems}
\phantomsection\label{\detokenize{dnpImport:dnpLab.dnpImport.vnmrj.import_fid}}\pysiglinewithargsret{\sphinxcode{\sphinxupquote{dnpLab.dnpImport.vnmrj.}}\sphinxbfcode{\sphinxupquote{import\_fid}}}{\emph{\DUrole{n}{path}}, \emph{\DUrole{n}{filename}\DUrole{o}{=}\DUrole{default_value}{\textquotesingle{}fid\textquotesingle{}}}}{}
Import VnmrJ fid file
\begin{quote}\begin{description}
\item[{Parameters}] \leavevmode\begin{itemize}
\item {} 
\sphinxstyleliteralstrong{\sphinxupquote{path}} (\sphinxstyleliteralemphasis{\sphinxupquote{str}}) \sphinxhyphen{}\sphinxhyphen{} Directory of fid file

\item {} 
\sphinxstyleliteralstrong{\sphinxupquote{filename}} (\sphinxstyleliteralemphasis{\sphinxupquote{str}}) \sphinxhyphen{}\sphinxhyphen{} Name of fid file. "fid" by default

\end{itemize}

\item[{Returns}] \leavevmode
Array of data

\item[{Return type}] \leavevmode
numpy.ndarray

\end{description}\end{quote}

\end{fulllineitems}

\index{import\_procpar() (in module dnpLab.dnpImport.vnmrj)@\spxentry{import\_procpar()}\spxextra{in module dnpLab.dnpImport.vnmrj}}

\begin{fulllineitems}
\phantomsection\label{\detokenize{dnpImport:dnpLab.dnpImport.vnmrj.import_procpar}}\pysiglinewithargsret{\sphinxcode{\sphinxupquote{dnpLab.dnpImport.vnmrj.}}\sphinxbfcode{\sphinxupquote{import\_procpar}}}{\emph{\DUrole{n}{path}}, \emph{\DUrole{n}{filename}\DUrole{o}{=}\DUrole{default_value}{\textquotesingle{}procpar\textquotesingle{}}}}{}
Import VnmrJ procpar parameters file
\begin{quote}\begin{description}
\item[{Parameters}] \leavevmode
\sphinxstyleliteralstrong{\sphinxupquote{path}} (\sphinxstyleliteralemphasis{\sphinxupquote{str}}) \sphinxhyphen{}\sphinxhyphen{} Directory of file

\item[{Returns}] \leavevmode
Dictionary of procpar parameters

\item[{Return type}] \leavevmode
dict

\end{description}\end{quote}

\end{fulllineitems}

\index{import\_vnmrj() (in module dnpLab.dnpImport.vnmrj)@\spxentry{import\_vnmrj()}\spxextra{in module dnpLab.dnpImport.vnmrj}}

\begin{fulllineitems}
\phantomsection\label{\detokenize{dnpImport:dnpLab.dnpImport.vnmrj.import_vnmrj}}\pysiglinewithargsret{\sphinxcode{\sphinxupquote{dnpLab.dnpImport.vnmrj.}}\sphinxbfcode{\sphinxupquote{import\_vnmrj}}}{\emph{\DUrole{n}{path}}, \emph{\DUrole{n}{fidFilename}\DUrole{o}{=}\DUrole{default_value}{\textquotesingle{}fid\textquotesingle{}}}, \emph{\DUrole{n}{paramFilename}\DUrole{o}{=}\DUrole{default_value}{\textquotesingle{}procpar\textquotesingle{}}}}{}
Import VnmrJ Data
\begin{quote}\begin{description}
\item[{Parameters}] \leavevmode\begin{itemize}
\item {} 
\sphinxstyleliteralstrong{\sphinxupquote{path}} (\sphinxstyleliteralemphasis{\sphinxupquote{str}}) \sphinxhyphen{}\sphinxhyphen{} path to experiment folder

\item {} 
\sphinxstyleliteralstrong{\sphinxupquote{fidFilename}} (\sphinxstyleliteralemphasis{\sphinxupquote{str}}) \sphinxhyphen{}\sphinxhyphen{} FID file name

\item {} 
\sphinxstyleliteralstrong{\sphinxupquote{paramFilename}} (\sphinxstyleliteralemphasis{\sphinxupquote{str}}) \sphinxhyphen{}\sphinxhyphen{} process parameter filename

\end{itemize}

\item[{Returns}] \leavevmode
data in dnpdata object

\item[{Return type}] \leavevmode
{\hyperref[\detokenize{dnpData:dnpLab.dnpdata}]{\sphinxcrossref{dnpdata}}}

\end{description}\end{quote}

\end{fulllineitems}



\subsection{Prospa Module}
\label{\detokenize{dnpImport:module-dnpLab.dnpImport.prospa}}\label{\detokenize{dnpImport:prospa-module}}\label{\detokenize{dnpImport:prospa}}\index{module@\spxentry{module}!dnpLab.dnpImport.prospa@\spxentry{dnpLab.dnpImport.prospa}}\index{dnpLab.dnpImport.prospa@\spxentry{dnpLab.dnpImport.prospa}!module@\spxentry{module}}\index{import\_prospa() (in module dnpLab.dnpImport.prospa)@\spxentry{import\_prospa()}\spxextra{in module dnpLab.dnpImport.prospa}}

\begin{fulllineitems}
\phantomsection\label{\detokenize{dnpImport:dnpLab.dnpImport.prospa.import_prospa}}\pysiglinewithargsret{\sphinxcode{\sphinxupquote{dnpLab.dnpImport.prospa.}}\sphinxbfcode{\sphinxupquote{import\_prospa}}}{\emph{\DUrole{n}{path}}, \emph{\DUrole{n}{parameters\_filename}\DUrole{o}{=}\DUrole{default_value}{None}}, \emph{\DUrole{n}{verbose}\DUrole{o}{=}\DUrole{default_value}{False}}}{}
Import Kea data
\begin{quote}\begin{description}
\item[{Parameters}] \leavevmode\begin{itemize}
\item {} 
\sphinxstyleliteralstrong{\sphinxupquote{path}} (\sphinxstyleliteralemphasis{\sphinxupquote{str}}) \sphinxhyphen{}\sphinxhyphen{} Path to data

\item {} 
\sphinxstyleliteralstrong{\sphinxupquote{num}} (\sphinxstyleliteralemphasis{\sphinxupquote{int}}) \sphinxhyphen{}\sphinxhyphen{} Experiment number

\item {} 
\sphinxstyleliteralstrong{\sphinxupquote{verbose}} (\sphinxstyleliteralemphasis{\sphinxupquote{bool}}) \sphinxhyphen{}\sphinxhyphen{} If true, prints additional information for troubleshooting

\end{itemize}

\item[{Returns}] \leavevmode
dnpdata object with Kea data

\end{description}\end{quote}

\end{fulllineitems}

\index{import\_prospa\_dir() (in module dnpLab.dnpImport.prospa)@\spxentry{import\_prospa\_dir()}\spxextra{in module dnpLab.dnpImport.prospa}}

\begin{fulllineitems}
\phantomsection\label{\detokenize{dnpImport:dnpLab.dnpImport.prospa.import_prospa_dir}}\pysiglinewithargsret{\sphinxcode{\sphinxupquote{dnpLab.dnpImport.prospa.}}\sphinxbfcode{\sphinxupquote{import\_prospa\_dir}}}{\emph{\DUrole{n}{path}}, \emph{\DUrole{n}{exp\_list}\DUrole{o}{=}\DUrole{default_value}{None}}}{}
Import directory of prospa experiments

\end{fulllineitems}

\index{import\_nd() (in module dnpLab.dnpImport.prospa)@\spxentry{import\_nd()}\spxextra{in module dnpLab.dnpImport.prospa}}

\begin{fulllineitems}
\phantomsection\label{\detokenize{dnpImport:dnpLab.dnpImport.prospa.import_nd}}\pysiglinewithargsret{\sphinxcode{\sphinxupquote{dnpLab.dnpImport.prospa.}}\sphinxbfcode{\sphinxupquote{import\_nd}}}{\emph{\DUrole{n}{path}}}{}
Import Kea 1d, 2d, 3d, 4d files
\begin{quote}\begin{description}
\item[{Parameters}] \leavevmode
\sphinxstyleliteralstrong{\sphinxupquote{path}} (\sphinxstyleliteralemphasis{\sphinxupquote{str}}) \sphinxhyphen{}\sphinxhyphen{} Path to file

\item[{Returns}] \leavevmode
x (None, numpy.array): Axes if included in binary file, None otherwise
data (numpy.array): Numpy array of data

\item[{Return type}] \leavevmode
tuple

\end{description}\end{quote}

\end{fulllineitems}

\index{import\_par() (in module dnpLab.dnpImport.prospa)@\spxentry{import\_par()}\spxextra{in module dnpLab.dnpImport.prospa}}

\begin{fulllineitems}
\phantomsection\label{\detokenize{dnpImport:dnpLab.dnpImport.prospa.import_par}}\pysiglinewithargsret{\sphinxcode{\sphinxupquote{dnpLab.dnpImport.prospa.}}\sphinxbfcode{\sphinxupquote{import\_par}}}{\emph{\DUrole{n}{path}}}{}
Import Kea parameters .par file
\begin{quote}\begin{description}
\item[{Parameters}] \leavevmode
\sphinxstyleliteralstrong{\sphinxupquote{path}} (\sphinxstyleliteralemphasis{\sphinxupquote{str}}) \sphinxhyphen{}\sphinxhyphen{} Path to parameters file

\item[{Returns}] \leavevmode
Dictionary of Kea Parameters

\item[{Return type}] \leavevmode
dict

\end{description}\end{quote}

\end{fulllineitems}

\index{import\_csv() (in module dnpLab.dnpImport.prospa)@\spxentry{import\_csv()}\spxextra{in module dnpLab.dnpImport.prospa}}

\begin{fulllineitems}
\phantomsection\label{\detokenize{dnpImport:dnpLab.dnpImport.prospa.import_csv}}\pysiglinewithargsret{\sphinxcode{\sphinxupquote{dnpLab.dnpImport.prospa.}}\sphinxbfcode{\sphinxupquote{import\_csv}}}{\emph{\DUrole{n}{path}}, \emph{\DUrole{n}{return\_raw}\DUrole{o}{=}\DUrole{default_value}{False}}, \emph{\DUrole{n}{is\_complex}\DUrole{o}{=}\DUrole{default_value}{True}}}{}
Import Kea csv file
\begin{quote}\begin{description}
\item[{Parameters}] \leavevmode
\sphinxstyleliteralstrong{\sphinxupquote{path}} (\sphinxstyleliteralemphasis{\sphinxupquote{str}}) \sphinxhyphen{}\sphinxhyphen{} Path to csv file

\item[{Returns}] \leavevmode
x(numpy.array): axes if return\_raw = False
data(numpy.array): Data in csv file

\item[{Return type}] \leavevmode
tuple

\end{description}\end{quote}

\end{fulllineitems}



\subsection{h5 Module}
\label{\detokenize{dnpImport:module-dnpLab.dnpImport.h5}}\label{\detokenize{dnpImport:h5-module}}\label{\detokenize{dnpImport:h5}}\index{module@\spxentry{module}!dnpLab.dnpImport.h5@\spxentry{dnpLab.dnpImport.h5}}\index{dnpLab.dnpImport.h5@\spxentry{dnpLab.dnpImport.h5}!module@\spxentry{module}}\index{saveh5() (in module dnpLab.dnpImport.h5)@\spxentry{saveh5()}\spxextra{in module dnpLab.dnpImport.h5}}

\begin{fulllineitems}
\phantomsection\label{\detokenize{dnpImport:dnpLab.dnpImport.h5.saveh5}}\pysiglinewithargsret{\sphinxcode{\sphinxupquote{dnpLab.dnpImport.h5.}}\sphinxbfcode{\sphinxupquote{saveh5}}}{\emph{\DUrole{n}{dataDict}}, \emph{\DUrole{n}{path}}, \emph{\DUrole{n}{overwrite}\DUrole{o}{=}\DUrole{default_value}{False}}}{}
Save workspace in .h5 format
\begin{quote}\begin{description}
\item[{Parameters}] \leavevmode\begin{itemize}
\item {} 
\sphinxstyleliteralstrong{\sphinxupquote{dataDict}} ({\hyperref[\detokenize{dnpData:dnpLab.dnpdata_collection}]{\sphinxcrossref{\sphinxstyleliteralemphasis{\sphinxupquote{dnpdata\_collection}}}}}) \sphinxhyphen{}\sphinxhyphen{} dnpdata\_collection object to save.

\item {} 
\sphinxstyleliteralstrong{\sphinxupquote{path}} (\sphinxstyleliteralemphasis{\sphinxupquote{str}}) \sphinxhyphen{}\sphinxhyphen{} Path to save data

\item {} 
\sphinxstyleliteralstrong{\sphinxupquote{overwrite}} (\sphinxstyleliteralemphasis{\sphinxupquote{bool}}) \sphinxhyphen{}\sphinxhyphen{} If True, h5 file can be overwritten. Otherwise, h5 file cannot be overwritten

\end{itemize}

\end{description}\end{quote}

\end{fulllineitems}

\index{write\_dnpdata() (in module dnpLab.dnpImport.h5)@\spxentry{write\_dnpdata()}\spxextra{in module dnpLab.dnpImport.h5}}

\begin{fulllineitems}
\phantomsection\label{\detokenize{dnpImport:dnpLab.dnpImport.h5.write_dnpdata}}\pysiglinewithargsret{\sphinxcode{\sphinxupquote{dnpLab.dnpImport.h5.}}\sphinxbfcode{\sphinxupquote{write\_dnpdata}}}{\emph{\DUrole{n}{dnpDataGroup}}, \emph{\DUrole{n}{dnpDataObject}}}{}
Takes file/group and writes dnpData object to it
\begin{quote}\begin{description}
\item[{Parameters}] \leavevmode\begin{itemize}
\item {} 
\sphinxstyleliteralstrong{\sphinxupquote{dnpDataGroup}} \sphinxhyphen{}\sphinxhyphen{} h5 group to save data to

\item {} 
\sphinxstyleliteralstrong{\sphinxupquote{dnpDataObject}} \sphinxhyphen{}\sphinxhyphen{} dnpdata object to save in h5 format

\end{itemize}

\end{description}\end{quote}

\end{fulllineitems}

\index{write\_dict() (in module dnpLab.dnpImport.h5)@\spxentry{write\_dict()}\spxextra{in module dnpLab.dnpImport.h5}}

\begin{fulllineitems}
\phantomsection\label{\detokenize{dnpImport:dnpLab.dnpImport.h5.write_dict}}\pysiglinewithargsret{\sphinxcode{\sphinxupquote{dnpLab.dnpImport.h5.}}\sphinxbfcode{\sphinxupquote{write\_dict}}}{\emph{\DUrole{n}{dnpDataGroup}}, \emph{\DUrole{n}{dnpDataObject}}}{}
Writes dictionary to h5 file

\end{fulllineitems}

\index{loadh5() (in module dnpLab.dnpImport.h5)@\spxentry{loadh5()}\spxextra{in module dnpLab.dnpImport.h5}}

\begin{fulllineitems}
\phantomsection\label{\detokenize{dnpImport:dnpLab.dnpImport.h5.loadh5}}\pysiglinewithargsret{\sphinxcode{\sphinxupquote{dnpLab.dnpImport.h5.}}\sphinxbfcode{\sphinxupquote{loadh5}}}{\emph{\DUrole{n}{path}}}{}
Returns Dictionary of dnpDataObjects
\begin{quote}\begin{description}
\item[{Parameters}] \leavevmode
\sphinxstyleliteralstrong{\sphinxupquote{path}} (\sphinxstyleliteralemphasis{\sphinxupquote{str}}) \sphinxhyphen{}\sphinxhyphen{} Path to h5 file

\item[{Returns}] \leavevmode
workspace object with data

\item[{Return type}] \leavevmode
{\hyperref[\detokenize{dnpData:dnpLab.dnpdata_collection}]{\sphinxcrossref{dnpdata\_collection}}}

\end{description}\end{quote}

\end{fulllineitems}



\section{dnpNMR}
\label{\detokenize{dnpNMR:dnpnmr}}\label{\detokenize{dnpNMR:id1}}\label{\detokenize{dnpNMR::doc}}

\subsection{Summary}
\label{\detokenize{dnpNMR:summary}}
The following table summarizes all available functions in this module


\subsection{Detailed Description of Functions}
\label{\detokenize{dnpNMR:module-dnpLab.dnpNMR}}\label{\detokenize{dnpNMR:detailed-description-of-functions}}\index{module@\spxentry{module}!dnpLab.dnpNMR@\spxentry{dnpLab.dnpNMR}}\index{dnpLab.dnpNMR@\spxentry{dnpLab.dnpNMR}!module@\spxentry{module}}\index{update\_parameters() (in module dnpLab.dnpNMR)@\spxentry{update\_parameters()}\spxextra{in module dnpLab.dnpNMR}}

\begin{fulllineitems}
\phantomsection\label{\detokenize{dnpNMR:dnpLab.dnpNMR.update_parameters}}\pysiglinewithargsret{\sphinxcode{\sphinxupquote{dnpLab.dnpNMR.}}\sphinxbfcode{\sphinxupquote{update\_parameters}}}{\emph{\DUrole{n}{proc\_parameters}}, \emph{\DUrole{n}{requiredList}}, \emph{\DUrole{n}{default\_parameters}}}{}
Add default parameter to processing parameters if a processing parameter is missing
\begin{quote}\begin{description}
\item[{Parameters}] \leavevmode\begin{itemize}
\item {} 
\sphinxstyleliteralstrong{\sphinxupquote{proc\_parameters}} (\sphinxstyleliteralemphasis{\sphinxupquote{dict}}) \sphinxhyphen{}\sphinxhyphen{} Dictionary of initial processing parameters

\item {} 
\sphinxstyleliteralstrong{\sphinxupquote{requiredList}} (\sphinxstyleliteralemphasis{\sphinxupquote{list}}) \sphinxhyphen{}\sphinxhyphen{} List of requrired processing parameters

\item {} 
\sphinxstyleliteralstrong{\sphinxupquote{default\_parameters}} (\sphinxstyleliteralemphasis{\sphinxupquote{dict}}) \sphinxhyphen{}\sphinxhyphen{} Dictionary of default processing parameters

\end{itemize}

\item[{Returns}] \leavevmode
Updated processing parameters dictionary

\item[{Return type}] \leavevmode
dict

\end{description}\end{quote}

\end{fulllineitems}

\index{remove\_offset() (in module dnpLab.dnpNMR)@\spxentry{remove\_offset()}\spxextra{in module dnpLab.dnpNMR}}

\begin{fulllineitems}
\phantomsection\label{\detokenize{dnpNMR:dnpLab.dnpNMR.remove_offset}}\pysiglinewithargsret{\sphinxcode{\sphinxupquote{dnpLab.dnpNMR.}}\sphinxbfcode{\sphinxupquote{remove\_offset}}}{\emph{\DUrole{n}{all\_data}}, \emph{\DUrole{n}{proc\_parameters}}}{}
Remove DC offset from FID by averaging the last few data points and subtracting the average
\begin{quote}\begin{description}
\item[{Parameters}] \leavevmode\begin{itemize}
\item {} 
\sphinxstyleliteralstrong{\sphinxupquote{all\_data}} ({\hyperref[\detokenize{dnpData:dnpLab.dnpdata}]{\sphinxcrossref{\sphinxstyleliteralemphasis{\sphinxupquote{dnpdata}}}}}\sphinxstyleliteralemphasis{\sphinxupquote{, }}\sphinxstyleliteralemphasis{\sphinxupquote{dict}}) \sphinxhyphen{}\sphinxhyphen{} Data container for data

\item {} 
\sphinxstyleliteralstrong{\sphinxupquote{proc\_parameters}} (\sphinxstyleliteralemphasis{\sphinxupquote{dict}}\sphinxstyleliteralemphasis{\sphinxupquote{,}}\sphinxstyleliteralemphasis{\sphinxupquote{procParam}}) \sphinxhyphen{}\sphinxhyphen{} Processing \_parameters

\end{itemize}

\end{description}\end{quote}


\begin{savenotes}\sphinxattablestart
\centering
\begin{tabulary}{\linewidth}[t]{|T|T|T|T|}
\hline

parameter
&
type
&
default
&
description
\\
\hline
dim
&
str
&
\textquotesingle{}t2\textquotesingle{}
&
Dimension to calculate DC offset
\\
\hline
offset\_points
&
int
&
10
&
Number of points at end of data to average for DC offset
\\
\hline
\end{tabulary}
\par
\sphinxattableend\end{savenotes}
\begin{quote}\begin{description}
\item[{Returns}] \leavevmode
If workspace is given returns dnpdata\_collection with data in processing buffer updated
dnpdata: If dnpdata object is given, return dnpdata object.

\item[{Return type}] \leavevmode
{\hyperref[\detokenize{dnpData:dnpLab.dnpdata_collection}]{\sphinxcrossref{dnpdata\_collection}}}

\end{description}\end{quote}

Example:

\begin{sphinxVerbatim}[commandchars=\\\{\}]
\PYG{n}{proc\PYGZus{}parameters} \PYG{o}{=} \PYG{p}{\PYGZob{}}\PYG{p}{\PYGZcb{}}
\PYG{n}{proc\PYGZus{}parameters}\PYG{p}{[}\PYG{l+s+s1}{\PYGZsq{}}\PYG{l+s+s1}{dim}\PYG{l+s+s1}{\PYGZsq{}}\PYG{p}{]} \PYG{o}{=} \PYG{l+s+s1}{\PYGZsq{}}\PYG{l+s+s1}{t2}\PYG{l+s+s1}{\PYGZsq{}}
\PYG{n}{proc\PYGZus{}parameters}\PYG{p}{[}\PYG{l+s+s1}{\PYGZsq{}}\PYG{l+s+s1}{offset\PYGZus{}points}\PYG{l+s+s1}{\PYGZsq{}}\PYG{p}{]} \PYG{o}{=} \PYG{l+m+mi}{10}

\PYG{n}{workspace} \PYG{o}{=} \PYG{n}{dnpLab}\PYG{o}{.}\PYG{n}{dnpNMR}\PYG{o}{.}\PYG{n}{remove\PYGZus{}offset}\PYG{p}{(}\PYG{n}{workspace}\PYG{p}{,} \PYG{n}{proc\PYGZus{}parameters}\PYG{p}{)}
\end{sphinxVerbatim}

\end{fulllineitems}

\index{fourier\_transform() (in module dnpLab.dnpNMR)@\spxentry{fourier\_transform()}\spxextra{in module dnpLab.dnpNMR}}

\begin{fulllineitems}
\phantomsection\label{\detokenize{dnpNMR:dnpLab.dnpNMR.fourier_transform}}\pysiglinewithargsret{\sphinxcode{\sphinxupquote{dnpLab.dnpNMR.}}\sphinxbfcode{\sphinxupquote{fourier\_transform}}}{\emph{\DUrole{n}{all\_data}}, \emph{\DUrole{n}{proc\_parameters}}}{}
Perform Fourier Transform down dim dimension given in proc\_parameters

\begin{sphinxadmonition}{note}{Note:}
Assumes dt = t{[}1{]} \sphinxhyphen{} t{[}0{]}
\end{sphinxadmonition}
\begin{quote}\begin{description}
\item[{Parameters}] \leavevmode\begin{itemize}
\item {} 
\sphinxstyleliteralstrong{\sphinxupquote{all\_data}} ({\hyperref[\detokenize{dnpData:dnpLab.dnpdata}]{\sphinxcrossref{\sphinxstyleliteralemphasis{\sphinxupquote{dnpdata}}}}}\sphinxstyleliteralemphasis{\sphinxupquote{, }}\sphinxstyleliteralemphasis{\sphinxupquote{dict}}) \sphinxhyphen{}\sphinxhyphen{} Data container

\item {} 
\sphinxstyleliteralstrong{\sphinxupquote{proc\_parameters}} (\sphinxstyleliteralemphasis{\sphinxupquote{dict}}\sphinxstyleliteralemphasis{\sphinxupquote{, }}\sphinxstyleliteralemphasis{\sphinxupquote{procParam}}) \sphinxhyphen{}\sphinxhyphen{} Processing parameters

\end{itemize}

\end{description}\end{quote}


\begin{savenotes}\sphinxattablestart
\centering
\begin{tabulary}{\linewidth}[t]{|T|T|T|T|}
\hline

parameter
&
type
&
default
&
description
\\
\hline
dim
&
str
&
\textquotesingle{}t2\textquotesingle{}
&
dimension to Fourier transform
\\
\hline
zero\_fill\_factor
&
int
&
2
&
factor to increase dim with zeros
\\
\hline
shift
&
bool
&
True
&
Perform fftshift to set zero frequency to center
\\
\hline
convert\_to\_ppm
&
bool
&
True
&
Convert dim from Hz to ppm
\\
\hline
\end{tabulary}
\par
\sphinxattableend\end{savenotes}
\begin{quote}\begin{description}
\item[{Returns}] \leavevmode
Processed data in container

\item[{Return type}] \leavevmode
all\_data ({\hyperref[\detokenize{dnpData:dnpLab.dnpdata}]{\sphinxcrossref{dnpdata}}}, dict)

\end{description}\end{quote}

Example:

\begin{sphinxVerbatim}[commandchars=\\\{\}]
\PYG{n}{proc\PYGZus{}parameters}\PYG{p}{[}\PYG{l+s+s1}{\PYGZsq{}}\PYG{l+s+s1}{dim}\PYG{l+s+s1}{\PYGZsq{}}\PYG{p}{]} \PYG{o}{=} \PYG{l+s+s1}{\PYGZsq{}}\PYG{l+s+s1}{t2}\PYG{l+s+s1}{\PYGZsq{}}
\PYG{n}{proc\PYGZus{}parameters}\PYG{p}{[}\PYG{l+s+s1}{\PYGZsq{}}\PYG{l+s+s1}{zero\PYGZus{}fill\PYGZus{}factor}\PYG{l+s+s1}{\PYGZsq{}}\PYG{p}{]} \PYG{o}{=} \PYG{l+m+mi}{2}
\PYG{n}{proc\PYGZus{}parameters}\PYG{p}{[}\PYG{l+s+s1}{\PYGZsq{}}\PYG{l+s+s1}{shift}\PYG{l+s+s1}{\PYGZsq{}}\PYG{p}{]} \PYG{o}{=} \PYG{k+kc}{True}
\PYG{n}{proc\PYGZus{}parameters}\PYG{p}{[}\PYG{l+s+s1}{\PYGZsq{}}\PYG{l+s+s1}{convert\PYGZus{}to\PYGZus{}ppm}\PYG{l+s+s1}{\PYGZsq{}}\PYG{p}{]} \PYG{o}{=} \PYG{k+kc}{True}

\PYG{n}{all\PYGZus{}data} \PYG{o}{=} \PYG{n}{dnpLab}\PYG{o}{.}\PYG{n}{dnpNMR}\PYG{o}{.}\PYG{n}{fourier\PYGZus{}transform}\PYG{p}{(}\PYG{n}{all\PYGZus{}data}\PYG{p}{,} \PYG{n}{proc\PYGZus{}parameters}\PYG{p}{)}
\end{sphinxVerbatim}

\end{fulllineitems}

\index{window() (in module dnpLab.dnpNMR)@\spxentry{window()}\spxextra{in module dnpLab.dnpNMR}}

\begin{fulllineitems}
\phantomsection\label{\detokenize{dnpNMR:dnpLab.dnpNMR.window}}\pysiglinewithargsret{\sphinxcode{\sphinxupquote{dnpLab.dnpNMR.}}\sphinxbfcode{\sphinxupquote{window}}}{\emph{\DUrole{n}{all\_data}}, \emph{\DUrole{n}{proc\_parameters}}}{}
Apply Apodization to data down given dimension
\begin{quote}\begin{description}
\item[{Parameters}] \leavevmode\begin{itemize}
\item {} 
\sphinxstyleliteralstrong{\sphinxupquote{all\_data}} ({\hyperref[\detokenize{dnpData:dnpLab.dnpdata}]{\sphinxcrossref{\sphinxstyleliteralemphasis{\sphinxupquote{dnpdata}}}}}\sphinxstyleliteralemphasis{\sphinxupquote{, }}\sphinxstyleliteralemphasis{\sphinxupquote{dict}}) \sphinxhyphen{}\sphinxhyphen{} data container

\item {} 
\sphinxstyleliteralstrong{\sphinxupquote{proc\_parameters}} (\sphinxstyleliteralemphasis{\sphinxupquote{dict}}\sphinxstyleliteralemphasis{\sphinxupquote{, }}\sphinxstyleliteralemphasis{\sphinxupquote{procParam}}) \sphinxhyphen{}\sphinxhyphen{} parameter values

\end{itemize}

\end{description}\end{quote}

\begin{sphinxadmonition}{note}{Note:}
Axis units assumed to be seconds
\end{sphinxadmonition}


\begin{savenotes}\sphinxattablestart
\centering
\begin{tabulary}{\linewidth}[t]{|T|T|T|T|}
\hline

parameter
&
type
&
default
&
description
\\
\hline
dim
&
str
&
\textquotesingle{}t2\textquotesingle{}
&
Dimension to apply exponential apodization
\\
\hline
linewidth
&
float
&
10
&
Linewidth of broadening to apply in Hz
\\
\hline
\end{tabulary}
\par
\sphinxattableend\end{savenotes}
\begin{quote}\begin{description}
\item[{Returns}] \leavevmode
data object with window function applied

\item[{Return type}] \leavevmode
{\hyperref[\detokenize{dnpData:dnpLab.dnpdata_collection}]{\sphinxcrossref{dnpdata\_collection}}} or {\hyperref[\detokenize{dnpData:dnpLab.dnpdata}]{\sphinxcrossref{dnpdata}}}

\end{description}\end{quote}

Example:

\begin{sphinxVerbatim}[commandchars=\\\{\}]
\PYG{n}{proc\PYGZus{}parameters} \PYG{o}{=} \PYG{p}{\PYGZob{}}
        \PYG{l+s+s1}{\PYGZsq{}}\PYG{l+s+s1}{linewidth}\PYG{l+s+s1}{\PYGZsq{}} \PYG{p}{:} \PYG{l+m+mi}{10}\PYG{p}{,}
        \PYG{l+s+s1}{\PYGZsq{}}\PYG{l+s+s1}{dim}\PYG{l+s+s1}{\PYGZsq{}} \PYG{p}{:} \PYG{l+s+s1}{\PYGZsq{}}\PYG{l+s+s1}{t2}\PYG{l+s+s1}{\PYGZsq{}}\PYG{p}{,}
        \PYG{p}{\PYGZcb{}}
\PYG{n}{all\PYGZus{}data} \PYG{o}{=} \PYG{n}{dnpLab}\PYG{o}{.}\PYG{n}{dnpNMR}\PYG{o}{.}\PYG{n}{window}\PYG{p}{(}\PYG{n}{all\PYGZus{}data}\PYG{p}{,}\PYG{n}{proc\PYGZus{}parameters}\PYG{p}{)}
\end{sphinxVerbatim}

\end{fulllineitems}

\index{integrate() (in module dnpLab.dnpNMR)@\spxentry{integrate()}\spxextra{in module dnpLab.dnpNMR}}

\begin{fulllineitems}
\phantomsection\label{\detokenize{dnpNMR:dnpLab.dnpNMR.integrate}}\pysiglinewithargsret{\sphinxcode{\sphinxupquote{dnpLab.dnpNMR.}}\sphinxbfcode{\sphinxupquote{integrate}}}{\emph{\DUrole{n}{all\_data}}, \emph{\DUrole{n}{proc\_parameters}}}{}
Integrate data down given dimension
\begin{quote}\begin{description}
\item[{Parameters}] \leavevmode\begin{itemize}
\item {} 
\sphinxstyleliteralstrong{\sphinxupquote{all\_data}} ({\hyperref[\detokenize{dnpData:dnpLab.dnpdata}]{\sphinxcrossref{\sphinxstyleliteralemphasis{\sphinxupquote{dnpdata}}}}}\sphinxstyleliteralemphasis{\sphinxupquote{,}}\sphinxstyleliteralemphasis{\sphinxupquote{dict}}) \sphinxhyphen{}\sphinxhyphen{} Data container

\item {} 
\sphinxstyleliteralstrong{\sphinxupquote{proc\_parameters}} (\sphinxstyleliteralemphasis{\sphinxupquote{dict}}\sphinxstyleliteralemphasis{\sphinxupquote{, }}\sphinxstyleliteralemphasis{\sphinxupquote{procParam}}) \sphinxhyphen{}\sphinxhyphen{} Processing Parameters

\end{itemize}

\end{description}\end{quote}


\begin{savenotes}\sphinxattablestart
\centering
\begin{tabulary}{\linewidth}[t]{|T|T|T|T|}
\hline

parameter
&
type
&
default
&
description
\\
\hline
dim
&
str
&
\textquotesingle{}t2\textquotesingle{}
&
dimension to integrate
\\
\hline
integrate\_center
&
float
&
0
&
center of integration window
\\
\hline
integrate\_width
&
float
&
100
&
width of integration window
\\
\hline
\end{tabulary}
\par
\sphinxattableend\end{savenotes}
\begin{quote}\begin{description}
\item[{Returns}] \leavevmode
Processed data

\item[{Return type}] \leavevmode
all\_data ({\hyperref[\detokenize{dnpData:dnpLab.dnpdata}]{\sphinxcrossref{dnpdata}}},dict)

\end{description}\end{quote}

Example:

\begin{sphinxVerbatim}[commandchars=\\\{\}]
\PYG{n}{proc\PYGZus{}parameters} \PYG{o}{=} \PYG{p}{\PYGZob{}}
    \PYG{l+s+s1}{\PYGZsq{}}\PYG{l+s+s1}{dim}\PYG{l+s+s1}{\PYGZsq{}} \PYG{p}{:} \PYG{l+s+s1}{\PYGZsq{}}\PYG{l+s+s1}{t2}\PYG{l+s+s1}{\PYGZsq{}}\PYG{p}{,}
    \PYG{l+s+s1}{\PYGZsq{}}\PYG{l+s+s1}{integrate\PYGZus{}center}\PYG{l+s+s1}{\PYGZsq{}} \PYG{p}{:} \PYG{l+m+mi}{0}\PYG{p}{,}
    \PYG{l+s+s1}{\PYGZsq{}}\PYG{l+s+s1}{integrate\PYGZus{}width}\PYG{l+s+s1}{\PYGZsq{}} \PYG{p}{:} \PYG{l+m+mi}{100}\PYG{p}{,}
    \PYG{p}{\PYGZcb{}}
\PYG{n}{dnpLab}\PYG{o}{.}\PYG{n}{dnpNMR}\PYG{o}{.}\PYG{n}{integrate}\PYG{p}{(}\PYG{n}{all\PYGZus{}data}\PYG{p}{,}\PYG{n}{proc\PYGZus{}parameters}\PYG{p}{)}
\end{sphinxVerbatim}

\end{fulllineitems}

\index{align() (in module dnpLab.dnpNMR)@\spxentry{align()}\spxextra{in module dnpLab.dnpNMR}}

\begin{fulllineitems}
\phantomsection\label{\detokenize{dnpNMR:dnpLab.dnpNMR.align}}\pysiglinewithargsret{\sphinxcode{\sphinxupquote{dnpLab.dnpNMR.}}\sphinxbfcode{\sphinxupquote{align}}}{\emph{\DUrole{n}{all\_data}}, \emph{\DUrole{n}{proc\_parameters}}}{}
Alignment of NMR spectra down given dim dimension

Example:

\begin{sphinxVerbatim}[commandchars=\\\{\}]
\PYG{n}{data} \PYG{o}{=} \PYG{n}{dnp}\PYG{o}{.}\PYG{n}{dnpNMR}\PYG{o}{.}\PYG{n}{align}\PYG{p}{(}\PYG{n}{data}\PYG{p}{,} \PYG{p}{\PYGZob{}}\PYG{p}{\PYGZcb{}}\PYG{p}{)}
\end{sphinxVerbatim}

\end{fulllineitems}

\index{autophase() (in module dnpLab.dnpNMR)@\spxentry{autophase()}\spxextra{in module dnpLab.dnpNMR}}

\begin{fulllineitems}
\phantomsection\label{\detokenize{dnpNMR:dnpLab.dnpNMR.autophase}}\pysiglinewithargsret{\sphinxcode{\sphinxupquote{dnpLab.dnpNMR.}}\sphinxbfcode{\sphinxupquote{autophase}}}{\emph{\DUrole{n}{workspace}}, \emph{\DUrole{n}{parameters}}}{}
Automatically phase data
\begin{quote}\begin{description}
\item[{Parameters}] \leavevmode\begin{itemize}
\item {} 
\sphinxstyleliteralstrong{\sphinxupquote{workspace}} ({\hyperref[\detokenize{dnpData:dnpLab.dnpdata_collection}]{\sphinxcrossref{\sphinxstyleliteralemphasis{\sphinxupquote{dnpdata\_collection}}}}}\sphinxstyleliteralemphasis{\sphinxupquote{, }}{\hyperref[\detokenize{dnpData:dnpLab.dnpdata}]{\sphinxcrossref{\sphinxstyleliteralemphasis{\sphinxupquote{dnpdata}}}}}) \sphinxhyphen{}\sphinxhyphen{} Data object to autophase

\item {} 
\sphinxstyleliteralstrong{\sphinxupquote{parameters}} (\sphinxstyleliteralemphasis{\sphinxupquote{dict}}) \sphinxhyphen{}\sphinxhyphen{} 

\end{itemize}

\item[{Returns}] \leavevmode
Autophased data

\item[{Return type}] \leavevmode
{\hyperref[\detokenize{dnpData:dnpLab.dnpdata_collection}]{\sphinxcrossref{dnpdata\_collection}}}, {\hyperref[\detokenize{dnpData:dnpLab.dnpdata}]{\sphinxcrossref{dnpdata}}}

\end{description}\end{quote}

Example:

\begin{sphinxVerbatim}[commandchars=\\\{\}]
\PYG{n}{ws} \PYG{o}{=} \PYG{n}{dnp}\PYG{o}{.}\PYG{n}{dnpNMR}\PYG{o}{.}\PYG{n}{autophase}\PYG{p}{(}\PYG{n}{ws}\PYG{p}{,} \PYG{p}{\PYGZob{}}\PYG{p}{\PYGZcb{}}\PYG{p}{)}
\end{sphinxVerbatim}

\end{fulllineitems}



\section{dnpFit}
\label{\detokenize{dnpFit:dnpfit}}\label{\detokenize{dnpFit::doc}}

\subsection{Summary}
\label{\detokenize{dnpFit:summary}}
The following table summarizes all available functions in this module


\subsection{Detailed Description of Functions}
\label{\detokenize{dnpFit:module-dnpLab.dnpFit}}\label{\detokenize{dnpFit:detailed-description-of-functions}}\index{module@\spxentry{module}!dnpLab.dnpFit@\spxentry{dnpLab.dnpFit}}\index{dnpLab.dnpFit@\spxentry{dnpLab.dnpFit}!module@\spxentry{module}}\index{t1Fit() (in module dnpLab.dnpFit)@\spxentry{t1Fit()}\spxextra{in module dnpLab.dnpFit}}

\begin{fulllineitems}
\phantomsection\label{\detokenize{dnpFit:dnpLab.dnpFit.t1Fit}}\pysiglinewithargsret{\sphinxcode{\sphinxupquote{dnpLab.dnpFit.}}\sphinxbfcode{\sphinxupquote{t1Fit}}}{\emph{\DUrole{n}{dataDict}}}{}
Fits inversion recovery data to extract T1 value in seconds
\begin{equation*}
\begin{split}f(t) = M_0 - M_{\infty} e^{-t/T_1}\end{split}
\end{equation*}\begin{quote}\begin{description}
\item[{Parameters}] \leavevmode
\sphinxstyleliteralstrong{\sphinxupquote{after processing inversion recovery data}}\sphinxstyleliteralstrong{\sphinxupquote{, }}\sphinxstyleliteralstrong{\sphinxupquote{after integration with dnpNMR.integrate}} (\sphinxstyleliteralemphasis{\sphinxupquote{workspace}}) \sphinxhyphen{}\sphinxhyphen{} 

\item[{Returns}] \leavevmode
Processed data in container, updated with fit data
attributes: T1 value and T1 standard deviation

\item[{Return type}] \leavevmode
all\_data ({\hyperref[\detokenize{dnpData:dnpLab.dnpdata}]{\sphinxcrossref{dnpdata}}}, dict)

\end{description}\end{quote}

Example:

\begin{sphinxVerbatim}[commandchars=\\\{\}]
\PYG{c+c1}{\PYGZsh{}\PYGZsh{}\PYGZsh{} INSERT importing and processing \PYGZsh{}\PYGZsh{}\PYGZsh{}}
\PYG{n}{dnpLab}\PYG{o}{.}\PYG{n}{dnpNMR}\PYG{o}{.}\PYG{n}{integrate}\PYG{p}{(}\PYG{n}{workspace}\PYG{p}{,} \PYG{p}{\PYGZob{}}\PYG{p}{\PYGZcb{}}\PYG{p}{)}

\PYG{n}{dnpLab}\PYG{o}{.}\PYG{n}{dnpFit}\PYG{o}{.}\PYG{n}{t1Fit}\PYG{p}{(}\PYG{n}{workspace}\PYG{p}{)}

\PYG{n}{T1\PYGZus{}value} \PYG{o}{=} \PYG{n}{workspace}\PYG{p}{[}\PYG{l+s+s1}{\PYGZsq{}}\PYG{l+s+s1}{fit}\PYG{l+s+s1}{\PYGZsq{}}\PYG{p}{]}\PYG{o}{.}\PYG{n}{attrs}\PYG{p}{[}\PYG{l+s+s1}{\PYGZsq{}}\PYG{l+s+s1}{t1}\PYG{l+s+s1}{\PYGZsq{}}\PYG{p}{]}
\PYG{n}{T1\PYGZus{}standard\PYGZus{}deviation} \PYG{o}{=} \PYG{n}{workspace}\PYG{p}{[}\PYG{l+s+s1}{\PYGZsq{}}\PYG{l+s+s1}{fit}\PYG{l+s+s1}{\PYGZsq{}}\PYG{p}{]}\PYG{o}{.}\PYG{n}{attrs}\PYG{p}{[}\PYG{l+s+s1}{\PYGZsq{}}\PYG{l+s+s1}{t1\PYGZus{}stdd}\PYG{l+s+s1}{\PYGZsq{}}\PYG{p}{]}
\PYG{n}{T1\PYGZus{}fit} \PYG{o}{=} \PYG{n}{workspace}\PYG{p}{[}\PYG{l+s+s1}{\PYGZsq{}}\PYG{l+s+s1}{fit}\PYG{l+s+s1}{\PYGZsq{}}\PYG{p}{]}\PYG{o}{.}\PYG{n}{values}
\PYG{n}{T1\PYGZus{}fit\PYGZus{}xaxis} \PYG{o}{=} \PYG{n}{workspace}\PYG{p}{[}\PYG{l+s+s1}{\PYGZsq{}}\PYG{l+s+s1}{fit}\PYG{l+s+s1}{\PYGZsq{}}\PYG{p}{]}\PYG{o}{.}\PYG{n}{coords}
\end{sphinxVerbatim}

\end{fulllineitems}

\index{enhancementFit() (in module dnpLab.dnpFit)@\spxentry{enhancementFit()}\spxextra{in module dnpLab.dnpFit}}

\begin{fulllineitems}
\phantomsection\label{\detokenize{dnpFit:dnpLab.dnpFit.enhancementFit}}\pysiglinewithargsret{\sphinxcode{\sphinxupquote{dnpLab.dnpFit.}}\sphinxbfcode{\sphinxupquote{enhancementFit}}}{\emph{\DUrole{n}{dataDict}}}{}
Fits enhancement curves to return Emax and power and one half maximum saturation
\begin{equation*}
\begin{split}f(p) = E_{max} p / (p_{1/2} + p)\end{split}
\end{equation*}\begin{quote}\begin{description}
\item[{Parameters}] \leavevmode
\sphinxstyleliteralstrong{\sphinxupquote{workspace}} \sphinxhyphen{}\sphinxhyphen{} 

\item[{Returns}] \leavevmode

Processed data in container, updated with fit data
attributes: Emax value and Emax standard deviation
\begin{quote}

p\_one\_half value and p\_one\_half standard deviation
\end{quote}


\item[{Return type}] \leavevmode
all\_data ({\hyperref[\detokenize{dnpData:dnpLab.dnpdata}]{\sphinxcrossref{dnpdata}}}, dict)

\end{description}\end{quote}

Example:

\begin{sphinxVerbatim}[commandchars=\\\{\}]
\PYG{c+c1}{\PYGZsh{}\PYGZsh{}\PYGZsh{} INSERT importing and processing \PYGZsh{}\PYGZsh{}\PYGZsh{}}
\PYG{n}{dnpLab}\PYG{o}{.}\PYG{n}{dnpNMR}\PYG{o}{.}\PYG{n}{integrate}\PYG{p}{(}\PYG{n}{workspace}\PYG{p}{,} \PYG{p}{\PYGZob{}}\PYG{p}{\PYGZcb{}}\PYG{p}{)}

\PYG{n}{workspace}\PYG{o}{.}\PYG{n}{new\PYGZus{}dim}\PYG{p}{(}\PYG{l+s+s1}{\PYGZsq{}}\PYG{l+s+s1}{power}\PYG{l+s+s1}{\PYGZsq{}}\PYG{p}{,} \PYG{n}{power\PYGZus{}list}\PYG{p}{)}

\PYG{n}{dnpLab}\PYG{o}{.}\PYG{n}{dnpFit}\PYG{o}{.}\PYG{n}{enhancementFit}\PYG{p}{(}\PYG{n}{workspace}\PYG{p}{)}

\PYG{n}{Emax\PYGZus{}value} \PYG{o}{=} \PYG{n}{workspace}\PYG{p}{[}\PYG{l+s+s1}{\PYGZsq{}}\PYG{l+s+s1}{fit}\PYG{l+s+s1}{\PYGZsq{}}\PYG{p}{]}\PYG{o}{.}\PYG{n}{attrs}\PYG{p}{[}\PYG{l+s+s1}{\PYGZsq{}}\PYG{l+s+s1}{E\PYGZus{}max}\PYG{l+s+s1}{\PYGZsq{}}\PYG{p}{]}
\PYG{n}{Emax\PYGZus{}standard\PYGZus{}deviation} \PYG{o}{=} \PYG{n}{workspace}\PYG{p}{[}\PYG{l+s+s1}{\PYGZsq{}}\PYG{l+s+s1}{fit}\PYG{l+s+s1}{\PYGZsq{}}\PYG{p}{]}\PYG{o}{.}\PYG{n}{attrs}\PYG{p}{[}\PYG{l+s+s1}{\PYGZsq{}}\PYG{l+s+s1}{E\PYGZus{}max\PYGZus{}stdd}\PYG{l+s+s1}{\PYGZsq{}}\PYG{p}{]}
\PYG{n}{p\PYGZus{}one\PYGZus{}half\PYGZus{}value} \PYG{o}{=} \PYG{n}{workspace}\PYG{p}{[}\PYG{l+s+s1}{\PYGZsq{}}\PYG{l+s+s1}{fit}\PYG{l+s+s1}{\PYGZsq{}}\PYG{p}{]}\PYG{o}{.}\PYG{n}{attrs}\PYG{p}{[}\PYG{l+s+s1}{\PYGZsq{}}\PYG{l+s+s1}{p\PYGZus{}half}\PYG{l+s+s1}{\PYGZsq{}}\PYG{p}{]}
\PYG{n}{p\PYGZus{}one\PYGZus{}half\PYGZus{}standard\PYGZus{}deviation} \PYG{o}{=} \PYG{n}{workspace}\PYG{p}{[}\PYG{l+s+s1}{\PYGZsq{}}\PYG{l+s+s1}{fit}\PYG{l+s+s1}{\PYGZsq{}}\PYG{p}{]}\PYG{o}{.}\PYG{n}{attrs}\PYG{p}{[}\PYG{l+s+s1}{\PYGZsq{}}\PYG{l+s+s1}{p\PYGZus{}half\PYGZus{}stdd}\PYG{l+s+s1}{\PYGZsq{}}\PYG{p}{]}
\PYG{n}{Emax\PYGZus{}fit} \PYG{o}{=} \PYG{n}{workspace}\PYG{p}{[}\PYG{l+s+s1}{\PYGZsq{}}\PYG{l+s+s1}{fit}\PYG{l+s+s1}{\PYGZsq{}}\PYG{p}{]}\PYG{o}{.}\PYG{n}{values}
\PYG{n}{Emax\PYGZus{}fit\PYGZus{}xaxis} \PYG{o}{=} \PYG{n}{workspace}\PYG{p}{[}\PYG{l+s+s1}{\PYGZsq{}}\PYG{l+s+s1}{fit}\PYG{l+s+s1}{\PYGZsq{}}\PYG{p}{]}\PYG{o}{.}\PYG{n}{coords}
\end{sphinxVerbatim}

\end{fulllineitems}



\section{dnpHydration}
\label{\detokenize{dnpHydration:dnphydration}}\label{\detokenize{dnpHydration::doc}}

\subsection{Summary}
\label{\detokenize{dnpHydration:summary}}
The following table summarizes all available functions in this module


\subsection{Detailed Description of Functions}
\label{\detokenize{dnpHydration:module-dnpLab.dnpHydration}}\label{\detokenize{dnpHydration:detailed-description-of-functions}}\index{module@\spxentry{module}!dnpLab.dnpHydration@\spxentry{dnpLab.dnpHydration}}\index{dnpLab.dnpHydration@\spxentry{dnpLab.dnpHydration}!module@\spxentry{module}}
dnpHydration module

This module calculates hydration related quantities using processed ODNP data.
\index{hydration() (in module dnpLab.dnpHydration)@\spxentry{hydration()}\spxextra{in module dnpLab.dnpHydration}}

\begin{fulllineitems}
\phantomsection\label{\detokenize{dnpHydration:dnpLab.dnpHydration.hydration}}\pysiglinewithargsret{\sphinxcode{\sphinxupquote{dnpLab.dnpHydration.}}\sphinxbfcode{\sphinxupquote{hydration}}}{\emph{\DUrole{n}{ws}}}{}
Calculating Hydration Results
\begin{quote}\begin{description}
\item[{Parameters}] \leavevmode
\sphinxstyleliteralstrong{\sphinxupquote{ws}} \sphinxhyphen{}\sphinxhyphen{} Workspace

\item[{Returns}] \leavevmode
A dictionary of hydration results

\item[{Return type}] \leavevmode
dict

\end{description}\end{quote}

\end{fulllineitems}

\index{HydrationParameter (class in dnpLab.dnpHydration)@\spxentry{HydrationParameter}\spxextra{class in dnpLab.dnpHydration}}

\begin{fulllineitems}
\phantomsection\label{\detokenize{dnpHydration:dnpLab.dnpHydration.HydrationParameter}}\pysigline{\sphinxbfcode{\sphinxupquote{class }}\sphinxcode{\sphinxupquote{dnpLab.dnpHydration.}}\sphinxbfcode{\sphinxupquote{HydrationParameter}}}
Hydration Parameters

Franck, JM, et. al.; "Anomalously Rapid Hydration Water Diffusion Dynamics Near DNA Surfaces" J. Am. Chem. Soc. 2015, 137, 12013−12023.
\index{ksigma\_bulk (dnpLab.dnpHydration.HydrationParameter attribute)@\spxentry{ksigma\_bulk}\spxextra{dnpLab.dnpHydration.HydrationParameter attribute}}

\begin{fulllineitems}
\phantomsection\label{\detokenize{dnpHydration:dnpLab.dnpHydration.HydrationParameter.ksigma_bulk}}\pysigline{\sphinxbfcode{\sphinxupquote{ksigma\_bulk}}\sphinxbfcode{\sphinxupquote{ = 95.4}}}
unit is s\textasciicircum{}\sphinxhyphen{}1 M\textasciicircum{}\sphinxhyphen{}1 (Figure 3 caption)
\begin{quote}\begin{description}
\item[{Type}] \leavevmode
float

\end{description}\end{quote}

\end{fulllineitems}

\index{klow\_bulk (dnpLab.dnpHydration.HydrationParameter attribute)@\spxentry{klow\_bulk}\spxextra{dnpLab.dnpHydration.HydrationParameter attribute}}

\begin{fulllineitems}
\phantomsection\label{\detokenize{dnpHydration:dnpLab.dnpHydration.HydrationParameter.klow_bulk}}\pysigline{\sphinxbfcode{\sphinxupquote{klow\_bulk}}\sphinxbfcode{\sphinxupquote{ = 366}}}
unit is s\textasciicircum{}\sphinxhyphen{}1 M\textasciicircum{}\sphinxhyphen{}1 (Figure 3 caption)
\begin{quote}\begin{description}
\item[{Type}] \leavevmode
float

\end{description}\end{quote}

\end{fulllineitems}

\index{tcorr\_bulk (dnpLab.dnpHydration.HydrationParameter attribute)@\spxentry{tcorr\_bulk}\spxextra{dnpLab.dnpHydration.HydrationParameter attribute}}

\begin{fulllineitems}
\phantomsection\label{\detokenize{dnpHydration:dnpLab.dnpHydration.HydrationParameter.tcorr_bulk}}\pysigline{\sphinxbfcode{\sphinxupquote{tcorr\_bulk}}\sphinxbfcode{\sphinxupquote{ = 54}}}
Corrected bulk tcorr, unit is ps, (section 2.5)
\begin{quote}\begin{description}
\item[{Type}] \leavevmode
float

\end{description}\end{quote}

\end{fulllineitems}

\index{D\_H2O (dnpLab.dnpHydration.HydrationParameter attribute)@\spxentry{D\_H2O}\spxextra{dnpLab.dnpHydration.HydrationParameter attribute}}

\begin{fulllineitems}
\phantomsection\label{\detokenize{dnpHydration:dnpLab.dnpHydration.HydrationParameter.D_H2O}}\pysigline{\sphinxbfcode{\sphinxupquote{D\_H2O}}\sphinxbfcode{\sphinxupquote{ = 2.3e\sphinxhyphen{}09}}}
(Eq. 19\sphinxhyphen{}20) bulk water diffusivity, unit is d\textasciicircum{}2/s where d is
distance in meters.
\begin{quote}\begin{description}
\item[{Type}] \leavevmode
float

\end{description}\end{quote}

\end{fulllineitems}

\index{D\_SL (dnpLab.dnpHydration.HydrationParameter attribute)@\spxentry{D\_SL}\spxextra{dnpLab.dnpHydration.HydrationParameter attribute}}

\begin{fulllineitems}
\phantomsection\label{\detokenize{dnpHydration:dnpLab.dnpHydration.HydrationParameter.D_SL}}\pysigline{\sphinxbfcode{\sphinxupquote{D\_SL}}\sphinxbfcode{\sphinxupquote{ = 4.1e\sphinxhyphen{}10}}}
(Eq. 19\sphinxhyphen{}20) spin label diffusivity, unit is d\textasciicircum{}2/s where d is
distance in meters.
\begin{quote}\begin{description}
\item[{Type}] \leavevmode
float

\end{description}\end{quote}

\end{fulllineitems}

\index{field (dnpLab.dnpHydration.HydrationParameter attribute)@\spxentry{field}\spxextra{dnpLab.dnpHydration.HydrationParameter attribute}}

\begin{fulllineitems}
\phantomsection\label{\detokenize{dnpHydration:dnpLab.dnpHydration.HydrationParameter.field}}\pysigline{\sphinxbfcode{\sphinxupquote{field}}\sphinxbfcode{\sphinxupquote{ = None}}}
Static magnetic field in mT, needed to find omega\_e and \_H
\begin{quote}\begin{description}
\item[{Type}] \leavevmode
float

\end{description}\end{quote}

\end{fulllineitems}

\index{spin\_C (dnpLab.dnpHydration.HydrationParameter attribute)@\spxentry{spin\_C}\spxextra{dnpLab.dnpHydration.HydrationParameter attribute}}

\begin{fulllineitems}
\phantomsection\label{\detokenize{dnpHydration:dnpLab.dnpHydration.HydrationParameter.spin_C}}\pysigline{\sphinxbfcode{\sphinxupquote{spin\_C}}\sphinxbfcode{\sphinxupquote{ = None}}}
(Eq. 1\sphinxhyphen{}2) unit is microM, spin label concentration for scaling
relaxations to get "relaxivities"
\begin{quote}\begin{description}
\item[{Type}] \leavevmode
float

\end{description}\end{quote}

\end{fulllineitems}

\index{T10 (dnpLab.dnpHydration.HydrationParameter attribute)@\spxentry{T10}\spxextra{dnpLab.dnpHydration.HydrationParameter attribute}}

\begin{fulllineitems}
\phantomsection\label{\detokenize{dnpHydration:dnpLab.dnpHydration.HydrationParameter.T10}}\pysigline{\sphinxbfcode{\sphinxupquote{T10}}\sphinxbfcode{\sphinxupquote{ = None}}}
T1 with spin label but at 0 mw E\_power, unit is sec
\begin{quote}\begin{description}
\item[{Type}] \leavevmode
float

\end{description}\end{quote}

\end{fulllineitems}

\index{T100 (dnpLab.dnpHydration.HydrationParameter attribute)@\spxentry{T100}\spxextra{dnpLab.dnpHydration.HydrationParameter attribute}}

\begin{fulllineitems}
\phantomsection\label{\detokenize{dnpHydration:dnpLab.dnpHydration.HydrationParameter.T100}}\pysigline{\sphinxbfcode{\sphinxupquote{T100}}\sphinxbfcode{\sphinxupquote{ = None}}}
T1 without spin label and without mw E\_power, unit is sec
\begin{quote}\begin{description}
\item[{Type}] \leavevmode
float

\end{description}\end{quote}

\end{fulllineitems}

\index{t1\_interp\_method() (dnpLab.dnpHydration.HydrationParameter property)@\spxentry{t1\_interp\_method()}\spxextra{dnpLab.dnpHydration.HydrationParameter property}}

\begin{fulllineitems}
\phantomsection\label{\detokenize{dnpHydration:dnpLab.dnpHydration.HydrationParameter.t1_interp_method}}\pysigline{\sphinxbfcode{\sphinxupquote{property }}\sphinxbfcode{\sphinxupquote{t1\_interp\_method}}}
Method used to interpolate T1, either \sphinxtitleref{linear} or \sphinxtitleref{second\_order}
\begin{quote}\begin{description}
\item[{Type}] \leavevmode
str

\end{description}\end{quote}

\end{fulllineitems}

\index{smax\_model() (dnpLab.dnpHydration.HydrationParameter property)@\spxentry{smax\_model()}\spxextra{dnpLab.dnpHydration.HydrationParameter property}}

\begin{fulllineitems}
\phantomsection\label{\detokenize{dnpHydration:dnpLab.dnpHydration.HydrationParameter.smax_model}}\pysigline{\sphinxbfcode{\sphinxupquote{property }}\sphinxbfcode{\sphinxupquote{smax\_model}}}
Method used to determine smax. Either \sphinxtitleref{tethered} or \sphinxtitleref{free}
\begin{quote}\begin{description}
\item[{Type}] \leavevmode
str

\end{description}\end{quote}

\end{fulllineitems}


\end{fulllineitems}

\index{HydrationCalculator (class in dnpLab.dnpHydration)@\spxentry{HydrationCalculator}\spxextra{class in dnpLab.dnpHydration}}

\begin{fulllineitems}
\phantomsection\label{\detokenize{dnpHydration:dnpLab.dnpHydration.HydrationCalculator}}\pysiglinewithargsret{\sphinxbfcode{\sphinxupquote{class }}\sphinxcode{\sphinxupquote{dnpLab.dnpHydration.}}\sphinxbfcode{\sphinxupquote{HydrationCalculator}}}{\emph{\DUrole{n}{T1}\DUrole{p}{:} \DUrole{n}{numpy.array}}, \emph{\DUrole{n}{T1\_power}\DUrole{p}{:} \DUrole{n}{numpy.array}}, \emph{\DUrole{n}{E}\DUrole{p}{:} \DUrole{n}{numpy.array}}, \emph{\DUrole{n}{E\_power}\DUrole{p}{:} \DUrole{n}{numpy.array}}, \emph{\DUrole{n}{hp}\DUrole{p}{:} \DUrole{n}{{\hyperref[\detokenize{dnpHydration:dnpLab.dnpHydration.HydrationParameter}]{\sphinxcrossref{dnpLab.dnpHydration.HydrationParameter}}}}}}{}
Bases: \sphinxcode{\sphinxupquote{object}}

Hydration Results Calculator
\index{T1 (dnpLab.dnpHydration.HydrationCalculator attribute)@\spxentry{T1}\spxextra{dnpLab.dnpHydration.HydrationCalculator attribute}}

\begin{fulllineitems}
\phantomsection\label{\detokenize{dnpHydration:dnpLab.dnpHydration.HydrationCalculator.T1}}\pysigline{\sphinxbfcode{\sphinxupquote{T1}}}
T1 array. Unit: second.
\begin{quote}\begin{description}
\item[{Type}] \leavevmode
numpy.array

\end{description}\end{quote}

\end{fulllineitems}

\index{T1\_power (dnpLab.dnpHydration.HydrationCalculator attribute)@\spxentry{T1\_power}\spxextra{dnpLab.dnpHydration.HydrationCalculator attribute}}

\begin{fulllineitems}
\phantomsection\label{\detokenize{dnpHydration:dnpLab.dnpHydration.HydrationCalculator.T1_power}}\pysigline{\sphinxbfcode{\sphinxupquote{T1\_power}}}
E\_power in Watt unit, same length as T1.
\begin{quote}\begin{description}
\item[{Type}] \leavevmode
numpy.array

\end{description}\end{quote}

\end{fulllineitems}

\index{E (dnpLab.dnpHydration.HydrationCalculator attribute)@\spxentry{E}\spxextra{dnpLab.dnpHydration.HydrationCalculator attribute}}

\begin{fulllineitems}
\phantomsection\label{\detokenize{dnpHydration:dnpLab.dnpHydration.HydrationCalculator.E}}\pysigline{\sphinxbfcode{\sphinxupquote{E}}}
Enhancements.
\begin{quote}\begin{description}
\item[{Type}] \leavevmode
numpy.array

\end{description}\end{quote}

\end{fulllineitems}

\index{E\_power (dnpLab.dnpHydration.HydrationCalculator attribute)@\spxentry{E\_power}\spxextra{dnpLab.dnpHydration.HydrationCalculator attribute}}

\begin{fulllineitems}
\phantomsection\label{\detokenize{dnpHydration:dnpLab.dnpHydration.HydrationCalculator.E_power}}\pysigline{\sphinxbfcode{\sphinxupquote{E\_power}}}
E\_power in Watt unit, same length as E.
\begin{quote}\begin{description}
\item[{Type}] \leavevmode
numpy.array

\end{description}\end{quote}

\end{fulllineitems}

\index{hp (dnpLab.dnpHydration.HydrationCalculator attribute)@\spxentry{hp}\spxextra{dnpLab.dnpHydration.HydrationCalculator attribute}}

\begin{fulllineitems}
\phantomsection\label{\detokenize{dnpHydration:dnpLab.dnpHydration.HydrationCalculator.hp}}\pysigline{\sphinxbfcode{\sphinxupquote{hp}}}
Parameters for calculation, including default
values.
\begin{quote}\begin{description}
\item[{Type}] \leavevmode
{\hyperref[\detokenize{dnpHydration:dnpLab.dnpHydration.HydrationParameter}]{\sphinxcrossref{HydrationParameter}}}

\end{description}\end{quote}

\end{fulllineitems}

\index{results (dnpLab.dnpHydration.HydrationCalculator attribute)@\spxentry{results}\spxextra{dnpLab.dnpHydration.HydrationCalculator attribute}}

\begin{fulllineitems}
\phantomsection\label{\detokenize{dnpHydration:dnpLab.dnpHydration.HydrationCalculator.results}}\pysigline{\sphinxbfcode{\sphinxupquote{results}}}
Hydration results.
\begin{quote}\begin{description}
\item[{Type}] \leavevmode
{\hyperref[\detokenize{dnpHydration:dnpLab.dnpHydration.HydrationResults}]{\sphinxcrossref{HydrationResults}}}

\end{description}\end{quote}

\end{fulllineitems}

\index{run() (dnpLab.dnpHydration.HydrationCalculator method)@\spxentry{run()}\spxextra{dnpLab.dnpHydration.HydrationCalculator method}}

\begin{fulllineitems}
\phantomsection\label{\detokenize{dnpHydration:dnpLab.dnpHydration.HydrationCalculator.run}}\pysiglinewithargsret{\sphinxbfcode{\sphinxupquote{run}}}{}{}
Run calculator

\end{fulllineitems}

\index{interpolate\_T1() (dnpLab.dnpHydration.HydrationCalculator method)@\spxentry{interpolate\_T1()}\spxextra{dnpLab.dnpHydration.HydrationCalculator method}}

\begin{fulllineitems}
\phantomsection\label{\detokenize{dnpHydration:dnpLab.dnpHydration.HydrationCalculator.interpolate_T1}}\pysiglinewithargsret{\sphinxbfcode{\sphinxupquote{interpolate\_T1}}}{\emph{\DUrole{n}{E\_power}\DUrole{p}{:} \DUrole{n}{numpy.array}}, \emph{\DUrole{n}{T1\_power}\DUrole{p}{:} \DUrole{n}{numpy.array}}, \emph{\DUrole{n}{T1}\DUrole{p}{:} \DUrole{n}{numpy.array}}}{}
Returns the one\sphinxhyphen{}dimensional piecewise interpolant to a function with
given discrete data points (T1\_power, T1), evaluated at E\_power.

Points outside the data range will be extrapolated
\begin{quote}\begin{description}
\item[{Parameters}] \leavevmode\begin{itemize}
\item {} 
\sphinxstyleliteralstrong{\sphinxupquote{E\_power}} \sphinxhyphen{}\sphinxhyphen{} The x\sphinxhyphen{}coordinates at which to evaluate.

\item {} 
\sphinxstyleliteralstrong{\sphinxupquote{T1\_power}} \sphinxhyphen{}\sphinxhyphen{} The x\sphinxhyphen{}coordinates of the data points, must be increasing.
Otherwise, T1\_power is internally sorted.

\item {} 
\sphinxstyleliteralstrong{\sphinxupquote{T1}} \sphinxhyphen{}\sphinxhyphen{} The y\sphinxhyphen{}coordinates of the data points, same length as T1\_power.

\end{itemize}

\item[{Returns}] \leavevmode
The evaluated values, same shape as E\_power.

\item[{Return type}] \leavevmode
interplatedT1 (np.array)

\end{description}\end{quote}

\end{fulllineitems}

\index{get\_tcorr() (dnpLab.dnpHydration.HydrationCalculator static method)@\spxentry{get\_tcorr()}\spxextra{dnpLab.dnpHydration.HydrationCalculator static method}}

\begin{fulllineitems}
\phantomsection\label{\detokenize{dnpHydration:dnpLab.dnpHydration.HydrationCalculator.get_tcorr}}\pysiglinewithargsret{\sphinxbfcode{\sphinxupquote{static }}\sphinxbfcode{\sphinxupquote{get\_tcorr}}}{\emph{\DUrole{n}{coupling\_factor}\DUrole{p}{:} \DUrole{n}{float}}, \emph{\DUrole{n}{omega\_e}\DUrole{p}{:} \DUrole{n}{float}}, \emph{\DUrole{n}{omega\_H}\DUrole{p}{:} \DUrole{n}{float}}}{}
Returns correlation time tcorr in pico second
\begin{quote}\begin{description}
\item[{Parameters}] \leavevmode\begin{itemize}
\item {} 
\sphinxstyleliteralstrong{\sphinxupquote{coupling\_factor}} (\sphinxstyleliteralemphasis{\sphinxupquote{float}}) \sphinxhyphen{}\sphinxhyphen{} 

\item {} 
\sphinxstyleliteralstrong{\sphinxupquote{omega\_e}} (\sphinxstyleliteralemphasis{\sphinxupquote{float}}) \sphinxhyphen{}\sphinxhyphen{} 

\item {} 
\sphinxstyleliteralstrong{\sphinxupquote{omega\_H}} (\sphinxstyleliteralemphasis{\sphinxupquote{float}}) \sphinxhyphen{}\sphinxhyphen{} 

\end{itemize}

\item[{Returns}] \leavevmode
correlation time in pico second

\item[{Return type}] \leavevmode
float

\item[{Raises}] \leavevmode
{\hyperref[\detokenize{dnpHydration:dnpLab.dnpHydration.FitError}]{\sphinxcrossref{\sphinxstyleliteralstrong{\sphinxupquote{FitError}}}}} \sphinxhyphen{}\sphinxhyphen{} If no available root is found.

\end{description}\end{quote}

\end{fulllineitems}

\index{get\_ksigma() (dnpLab.dnpHydration.HydrationCalculator static method)@\spxentry{get\_ksigma()}\spxextra{dnpLab.dnpHydration.HydrationCalculator static method}}

\begin{fulllineitems}
\phantomsection\label{\detokenize{dnpHydration:dnpLab.dnpHydration.HydrationCalculator.get_ksigma}}\pysiglinewithargsret{\sphinxbfcode{\sphinxupquote{static }}\sphinxbfcode{\sphinxupquote{get\_ksigma}}}{\emph{\DUrole{n}{ksig\_sp}\DUrole{p}{:} \DUrole{n}{numpy.array}}, \emph{\DUrole{n}{power}\DUrole{p}{:} \DUrole{n}{numpy.array}}}{}
Get ksigma and E\_power at half max of ksig
\begin{quote}\begin{description}
\item[{Parameters}] \leavevmode\begin{itemize}
\item {} 
\sphinxstyleliteralstrong{\sphinxupquote{ksig}} (\sphinxstyleliteralemphasis{\sphinxupquote{numpy.array}}) \sphinxhyphen{}\sphinxhyphen{} Array of ksigma.

\item {} 
\sphinxstyleliteralstrong{\sphinxupquote{power}} (\sphinxstyleliteralemphasis{\sphinxupquote{numpy.array}}) \sphinxhyphen{}\sphinxhyphen{} Array of E\_power.

\end{itemize}

\item[{Returns}] \leavevmode
fit results
pcov: covariance matrix

\item[{Return type}] \leavevmode
popt

\end{description}\end{quote}
\begin{description}
\item[{Asserts:}] \leavevmode
ksigma (popt{[}0{]}) is greater than zero

\end{description}

\end{fulllineitems}

\index{get\_uncorrected\_xi() (dnpLab.dnpHydration.HydrationCalculator static method)@\spxentry{get\_uncorrected\_xi()}\spxextra{dnpLab.dnpHydration.HydrationCalculator static method}}

\begin{fulllineitems}
\phantomsection\label{\detokenize{dnpHydration:dnpLab.dnpHydration.HydrationCalculator.get_uncorrected_xi}}\pysiglinewithargsret{\sphinxbfcode{\sphinxupquote{static }}\sphinxbfcode{\sphinxupquote{get\_uncorrected\_xi}}}{\emph{\DUrole{n}{Ep}\DUrole{p}{:} \DUrole{n}{numpy.array}}, \emph{\DUrole{n}{power}\DUrole{p}{:} \DUrole{n}{numpy.array}}, \emph{\DUrole{n}{T10}\DUrole{p}{:} \DUrole{n}{float}}, \emph{\DUrole{n}{T100}\DUrole{p}{:} \DUrole{n}{float}}, \emph{\DUrole{n}{wRatio}\DUrole{p}{:} \DUrole{n}{float}}, \emph{\DUrole{n}{s\_max}\DUrole{p}{:} \DUrole{n}{float}}}{}
Get coupling\_factor and E\_power at half saturation
\begin{quote}\begin{description}
\item[{Parameters}] \leavevmode\begin{itemize}
\item {} 
\sphinxstyleliteralstrong{\sphinxupquote{Ep}} (\sphinxstyleliteralemphasis{\sphinxupquote{numpy.array}}) \sphinxhyphen{}\sphinxhyphen{} Array of enhancements.

\item {} 
\sphinxstyleliteralstrong{\sphinxupquote{power}} (\sphinxstyleliteralemphasis{\sphinxupquote{numpy.array}}) \sphinxhyphen{}\sphinxhyphen{} Array of E\_power.

\item {} 
\sphinxstyleliteralstrong{\sphinxupquote{T10}} (\sphinxstyleliteralemphasis{\sphinxupquote{float}}) \sphinxhyphen{}\sphinxhyphen{} T10

\item {} 
\sphinxstyleliteralstrong{\sphinxupquote{T100}} (\sphinxstyleliteralemphasis{\sphinxupquote{float}}) \sphinxhyphen{}\sphinxhyphen{} T100

\item {} 
\sphinxstyleliteralstrong{\sphinxupquote{wRatio}} (\sphinxstyleliteralemphasis{\sphinxupquote{float}}) \sphinxhyphen{}\sphinxhyphen{} ratio of electron \& proton Larmor frequencies

\item {} 
\sphinxstyleliteralstrong{\sphinxupquote{s\_max}} (\sphinxstyleliteralemphasis{\sphinxupquote{float}}) \sphinxhyphen{}\sphinxhyphen{} maximal saturation factor

\end{itemize}

\item[{Returns}] \leavevmode
A tuple of float (coupling\_factor, p\_12).

\item[{Raises}] \leavevmode
{\hyperref[\detokenize{dnpHydration:dnpLab.dnpHydration.FitError}]{\sphinxcrossref{\sphinxstyleliteralstrong{\sphinxupquote{FitError}}}}} \sphinxhyphen{}\sphinxhyphen{} If least square fitting is not succeed.

\end{description}\end{quote}

\end{fulllineitems}


\end{fulllineitems}

\index{HydrationResults (class in dnpLab.dnpHydration)@\spxentry{HydrationResults}\spxextra{class in dnpLab.dnpHydration}}

\begin{fulllineitems}
\phantomsection\label{\detokenize{dnpHydration:dnpLab.dnpHydration.HydrationResults}}\pysiglinewithargsret{\sphinxbfcode{\sphinxupquote{class }}\sphinxcode{\sphinxupquote{dnpLab.dnpHydration.}}\sphinxbfcode{\sphinxupquote{HydrationResults}}}{\emph{\DUrole{o}{*}\DUrole{n}{args}}, \emph{\DUrole{o}{**}\DUrole{n}{kwargs}}}{}
Bases: {\hyperref[\detokenize{dnpHydration:dnpLab.dnpHydration.AttrDict}]{\sphinxcrossref{\sphinxcode{\sphinxupquote{dnpLab.dnpHydration.AttrDict}}}}}

Class for handling hydration related quantities
\index{uncorrected\_Ep (dnpLab.dnpHydration.HydrationResults attribute)@\spxentry{uncorrected\_Ep}\spxextra{dnpLab.dnpHydration.HydrationResults attribute}}

\begin{fulllineitems}
\phantomsection\label{\detokenize{dnpHydration:dnpLab.dnpHydration.HydrationResults.uncorrected_Ep}}\pysigline{\sphinxbfcode{\sphinxupquote{uncorrected\_Ep}}}
Fit of Ep array
\begin{quote}\begin{description}
\item[{Type}] \leavevmode
numpy.array

\end{description}\end{quote}

\end{fulllineitems}

\index{interpolated\_T1 (dnpLab.dnpHydration.HydrationResults attribute)@\spxentry{interpolated\_T1}\spxextra{dnpLab.dnpHydration.HydrationResults attribute}}

\begin{fulllineitems}
\phantomsection\label{\detokenize{dnpHydration:dnpLab.dnpHydration.HydrationResults.interpolated_T1}}\pysigline{\sphinxbfcode{\sphinxupquote{interpolated\_T1}}}
T1 values interpolated on E\_power,
\begin{quote}\begin{description}
\item[{Type}] \leavevmode
numpy.array

\end{description}\end{quote}

\end{fulllineitems}

\index{ksigma\_array (dnpLab.dnpHydration.HydrationResults attribute)@\spxentry{ksigma\_array}\spxextra{dnpLab.dnpHydration.HydrationResults attribute}}

\begin{fulllineitems}
\phantomsection\label{\detokenize{dnpHydration:dnpLab.dnpHydration.HydrationResults.ksigma_array}}\pysigline{\sphinxbfcode{\sphinxupquote{ksigma\_array}}}
numpy array that is the result of \textasciitilde{}(1\sphinxhyphen{}E) / {[} (constants*T1) {]},
used in ksigma(E\_power) fit,
\begin{quote}\begin{description}
\item[{Type}] \leavevmode
numpy.array

\end{description}\end{quote}

\end{fulllineitems}

\index{ksigma\_fit (dnpLab.dnpHydration.HydrationResults attribute)@\spxentry{ksigma\_fit}\spxextra{dnpLab.dnpHydration.HydrationResults attribute}}

\begin{fulllineitems}
\phantomsection\label{\detokenize{dnpHydration:dnpLab.dnpHydration.HydrationResults.ksigma_fit}}\pysigline{\sphinxbfcode{\sphinxupquote{ksigma\_fit}}}
ksig\_fit,
\begin{quote}\begin{description}
\item[{Type}] \leavevmode
numpy.array

\end{description}\end{quote}

\end{fulllineitems}

\index{ksigma (dnpLab.dnpHydration.HydrationResults attribute)@\spxentry{ksigma}\spxextra{dnpLab.dnpHydration.HydrationResults attribute}}

\begin{fulllineitems}
\phantomsection\label{\detokenize{dnpHydration:dnpLab.dnpHydration.HydrationResults.ksigma}}\pysigline{\sphinxbfcode{\sphinxupquote{ksigma}}}
ksigma,
\begin{quote}\begin{description}
\item[{Type}] \leavevmode
float

\end{description}\end{quote}

\end{fulllineitems}

\index{ksigma\_stdd (dnpLab.dnpHydration.HydrationResults attribute)@\spxentry{ksigma\_stdd}\spxextra{dnpLab.dnpHydration.HydrationResults attribute}}

\begin{fulllineitems}
\phantomsection\label{\detokenize{dnpHydration:dnpLab.dnpHydration.HydrationResults.ksigma_stdd}}\pysigline{\sphinxbfcode{\sphinxupquote{ksigma\_stdd}}}
ksigma\_stdd,
\begin{quote}\begin{description}
\item[{Type}] \leavevmode
float

\end{description}\end{quote}

\end{fulllineitems}

\index{ksigma\_bulk\_ratio (dnpLab.dnpHydration.HydrationResults attribute)@\spxentry{ksigma\_bulk\_ratio}\spxextra{dnpLab.dnpHydration.HydrationResults attribute}}

\begin{fulllineitems}
\phantomsection\label{\detokenize{dnpHydration:dnpLab.dnpHydration.HydrationResults.ksigma_bulk_ratio}}\pysigline{\sphinxbfcode{\sphinxupquote{ksigma\_bulk\_ratio}}}
ksigma/ksigma\_bulk,
\begin{quote}\begin{description}
\item[{Type}] \leavevmode
float

\end{description}\end{quote}

\end{fulllineitems}

\index{krho (dnpLab.dnpHydration.HydrationResults attribute)@\spxentry{krho}\spxextra{dnpLab.dnpHydration.HydrationResults attribute}}

\begin{fulllineitems}
\phantomsection\label{\detokenize{dnpHydration:dnpLab.dnpHydration.HydrationResults.krho}}\pysigline{\sphinxbfcode{\sphinxupquote{krho}}}
krho,
\begin{quote}\begin{description}
\item[{Type}] \leavevmode
float

\end{description}\end{quote}

\end{fulllineitems}

\index{klow (dnpLab.dnpHydration.HydrationResults attribute)@\spxentry{klow}\spxextra{dnpLab.dnpHydration.HydrationResults attribute}}

\begin{fulllineitems}
\phantomsection\label{\detokenize{dnpHydration:dnpLab.dnpHydration.HydrationResults.klow}}\pysigline{\sphinxbfcode{\sphinxupquote{klow}}}
klow,
\begin{quote}\begin{description}
\item[{Type}] \leavevmode
float

\end{description}\end{quote}

\end{fulllineitems}

\index{klow\_bulk\_ratio (dnpLab.dnpHydration.HydrationResults attribute)@\spxentry{klow\_bulk\_ratio}\spxextra{dnpLab.dnpHydration.HydrationResults attribute}}

\begin{fulllineitems}
\phantomsection\label{\detokenize{dnpHydration:dnpLab.dnpHydration.HydrationResults.klow_bulk_ratio}}\pysigline{\sphinxbfcode{\sphinxupquote{klow\_bulk\_ratio}}}
klow / klow\_bulk,
\begin{quote}\begin{description}
\item[{Type}] \leavevmode
float

\end{description}\end{quote}

\end{fulllineitems}

\index{coupling\_factor (dnpLab.dnpHydration.HydrationResults attribute)@\spxentry{coupling\_factor}\spxextra{dnpLab.dnpHydration.HydrationResults attribute}}

\begin{fulllineitems}
\phantomsection\label{\detokenize{dnpHydration:dnpLab.dnpHydration.HydrationResults.coupling_factor}}\pysigline{\sphinxbfcode{\sphinxupquote{coupling\_factor}}}
coupling\_factor,
\begin{quote}\begin{description}
\item[{Type}] \leavevmode
float

\end{description}\end{quote}

\end{fulllineitems}

\index{tcorr (dnpLab.dnpHydration.HydrationResults attribute)@\spxentry{tcorr}\spxextra{dnpLab.dnpHydration.HydrationResults attribute}}

\begin{fulllineitems}
\phantomsection\label{\detokenize{dnpHydration:dnpLab.dnpHydration.HydrationResults.tcorr}}\pysigline{\sphinxbfcode{\sphinxupquote{tcorr}}}
tcorr,
\begin{quote}\begin{description}
\item[{Type}] \leavevmode
float

\end{description}\end{quote}

\end{fulllineitems}

\index{tcorr\_bulk\_ratio (dnpLab.dnpHydration.HydrationResults attribute)@\spxentry{tcorr\_bulk\_ratio}\spxextra{dnpLab.dnpHydration.HydrationResults attribute}}

\begin{fulllineitems}
\phantomsection\label{\detokenize{dnpHydration:dnpLab.dnpHydration.HydrationResults.tcorr_bulk_ratio}}\pysigline{\sphinxbfcode{\sphinxupquote{tcorr\_bulk\_ratio}}}
tcorr / tcorr\_bulk,
\begin{quote}\begin{description}
\item[{Type}] \leavevmode
float

\end{description}\end{quote}

\end{fulllineitems}

\index{Dlocal (dnpLab.dnpHydration.HydrationResults attribute)@\spxentry{Dlocal}\spxextra{dnpLab.dnpHydration.HydrationResults attribute}}

\begin{fulllineitems}
\phantomsection\label{\detokenize{dnpHydration:dnpLab.dnpHydration.HydrationResults.Dlocal}}\pysigline{\sphinxbfcode{\sphinxupquote{Dlocal}}}
Dlocal
\begin{quote}\begin{description}
\item[{Type}] \leavevmode
float

\end{description}\end{quote}

\end{fulllineitems}


\end{fulllineitems}

\index{FitError (class in dnpLab.dnpHydration)@\spxentry{FitError}\spxextra{class in dnpLab.dnpHydration}}

\begin{fulllineitems}
\phantomsection\label{\detokenize{dnpHydration:dnpLab.dnpHydration.FitError}}\pysigline{\sphinxbfcode{\sphinxupquote{class }}\sphinxcode{\sphinxupquote{dnpLab.dnpHydration.}}\sphinxbfcode{\sphinxupquote{FitError}}}
Bases: \sphinxcode{\sphinxupquote{Exception}}

Exception of Failed Fitting

\end{fulllineitems}

\index{AttrDict (class in dnpLab.dnpHydration)@\spxentry{AttrDict}\spxextra{class in dnpLab.dnpHydration}}

\begin{fulllineitems}
\phantomsection\label{\detokenize{dnpHydration:dnpLab.dnpHydration.AttrDict}}\pysiglinewithargsret{\sphinxbfcode{\sphinxupquote{class }}\sphinxcode{\sphinxupquote{dnpLab.dnpHydration.}}\sphinxbfcode{\sphinxupquote{AttrDict}}}{\emph{\DUrole{o}{*}\DUrole{n}{args}}, \emph{\DUrole{o}{**}\DUrole{n}{kwargs}}}{}
Bases: \sphinxcode{\sphinxupquote{object}}

Class with Dictionary\sphinxhyphen{}like Setting and Getting
\index{update() (dnpLab.dnpHydration.AttrDict method)@\spxentry{update()}\spxextra{dnpLab.dnpHydration.AttrDict method}}

\begin{fulllineitems}
\phantomsection\label{\detokenize{dnpHydration:dnpLab.dnpHydration.AttrDict.update}}\pysiglinewithargsret{\sphinxbfcode{\sphinxupquote{update}}}{\emph{\DUrole{n}{init}\DUrole{o}{=}\DUrole{default_value}{None}}, \emph{\DUrole{o}{**}\DUrole{n}{kwargs}}}{}
Update existing parameters
\begin{quote}\begin{description}
\item[{Parameters}] \leavevmode
\sphinxstyleliteralstrong{\sphinxupquote{init}} \sphinxhyphen{}\sphinxhyphen{} If init is present and has a .keys() method,
then does: for k in init: D{[}k{]} = E{[}k{]}.
If init is present and lacks a .keys() method,
then does: for k, v in init: D{[}k{]} = v

\end{description}\end{quote}

\end{fulllineitems}


\end{fulllineitems}



\section{dnpResults}
\label{\detokenize{dnpResults:dnpresults}}\label{\detokenize{dnpResults::doc}}

\subsection{Summary}
\label{\detokenize{dnpResults:summary}}
The following table summarizes all available functions in this module


\subsection{Detailed Description of Functions}
\label{\detokenize{dnpResults:module-dnpLab.dnpResults}}\label{\detokenize{dnpResults:detailed-description-of-functions}}\index{module@\spxentry{module}!dnpLab.dnpResults@\spxentry{dnpLab.dnpResults}}\index{dnpLab.dnpResults@\spxentry{dnpLab.dnpResults}!module@\spxentry{module}}\index{imshow() (in module dnpLab.dnpResults)@\spxentry{imshow()}\spxextra{in module dnpLab.dnpResults}}

\begin{fulllineitems}
\phantomsection\label{\detokenize{dnpResults:dnpLab.dnpResults.imshow}}\pysiglinewithargsret{\sphinxcode{\sphinxupquote{dnpLab.dnpResults.}}\sphinxbfcode{\sphinxupquote{imshow}}}{\emph{\DUrole{n}{data}}, \emph{\DUrole{o}{*}\DUrole{n}{args}}, \emph{\DUrole{o}{**}\DUrole{n}{kwargs}}}{}
Image Plot for dnpdata object
\begin{quote}\begin{description}
\item[{Parameters}] \leavevmode\begin{itemize}
\item {} 
\sphinxstyleliteralstrong{\sphinxupquote{data}} ({\hyperref[\detokenize{dnpData:dnpLab.dnpdata}]{\sphinxcrossref{\sphinxstyleliteralemphasis{\sphinxupquote{dnpdata}}}}}) \sphinxhyphen{}\sphinxhyphen{} dnpdata object for image plot

\item {} 
\sphinxstyleliteralstrong{\sphinxupquote{args}} \sphinxhyphen{}\sphinxhyphen{} args for matplotlib imshow function

\item {} 
\sphinxstyleliteralstrong{\sphinxupquote{kwargs}} \sphinxhyphen{}\sphinxhyphen{} kwargs for matplotlib imshow function

\end{itemize}

\end{description}\end{quote}

Example:

\begin{sphinxVerbatim}[commandchars=\\\{\}]
\PYG{c+c1}{\PYGZsh{} Plotting a dnpdata object}
\PYG{n}{dnp}\PYG{o}{.}\PYG{n}{dnpResults}\PYG{o}{.}\PYG{n}{plt}\PYG{o}{.}\PYG{n}{figure}\PYG{p}{(}\PYG{p}{)}
\PYG{n}{dnp}\PYG{o}{.}\PYG{n}{dnpResults}\PYG{o}{.}\PYG{n}{imshow}\PYG{p}{(}\PYG{n}{data}\PYG{p}{)}
\PYG{n}{dnp}\PYG{o}{.}\PYG{n}{dnpResults}\PYG{o}{.}\PYG{n}{plt}\PYG{o}{.}\PYG{n}{show}\PYG{p}{(}\PYG{p}{)}

\PYG{c+c1}{\PYGZsh{} Plotting a workspace (dnpdata\PYGZus{}collection)}
\PYG{n}{dnp}\PYG{o}{.}\PYG{n}{dnpResults}\PYG{o}{.}\PYG{n}{plt}\PYG{o}{.}\PYG{n}{figure}\PYG{p}{(}\PYG{p}{)}
\PYG{n}{dnp}\PYG{o}{.}\PYG{n}{dnpResults}\PYG{o}{.}\PYG{n}{imshow}\PYG{p}{(}\PYG{n}{ws}\PYG{p}{[}\PYG{l+s+s1}{\PYGZsq{}}\PYG{l+s+s1}{proc}\PYG{l+s+s1}{\PYGZsq{}}\PYG{p}{]}\PYG{p}{)}
\PYG{n}{dnp}\PYG{o}{.}\PYG{n}{dnpResults}\PYG{o}{.}\PYG{n}{plt}\PYG{o}{.}\PYG{n}{show}\PYG{p}{(}\PYG{p}{)}
\end{sphinxVerbatim}

\end{fulllineitems}

\index{plot() (in module dnpLab.dnpResults)@\spxentry{plot()}\spxextra{in module dnpLab.dnpResults}}

\begin{fulllineitems}
\phantomsection\label{\detokenize{dnpResults:dnpLab.dnpResults.plot}}\pysiglinewithargsret{\sphinxcode{\sphinxupquote{dnpLab.dnpResults.}}\sphinxbfcode{\sphinxupquote{plot}}}{\emph{\DUrole{n}{data}}, \emph{\DUrole{o}{*}\DUrole{n}{args}}, \emph{\DUrole{o}{**}\DUrole{n}{kwargs}}}{}
Plot function for dnpdata object
\begin{quote}\begin{description}
\item[{Parameters}] \leavevmode\begin{itemize}
\item {} 
\sphinxstyleliteralstrong{\sphinxupquote{data}} ({\hyperref[\detokenize{dnpData:dnpLab.dnpdata}]{\sphinxcrossref{\sphinxstyleliteralemphasis{\sphinxupquote{dnpdata}}}}}) \sphinxhyphen{}\sphinxhyphen{} dnpdata object for matplotlib plot function

\item {} 
\sphinxstyleliteralstrong{\sphinxupquote{args}} \sphinxhyphen{}\sphinxhyphen{} args for matplotlib plot function

\item {} 
\sphinxstyleliteralstrong{\sphinxupquote{kwargs}} \sphinxhyphen{}\sphinxhyphen{} kwargs for matplotlib plot function

\end{itemize}

\end{description}\end{quote}

Example:

\begin{sphinxVerbatim}[commandchars=\\\{\}]
\PYG{c+c1}{\PYGZsh{} Plotting a dnpdata object}
\PYG{n}{dnp}\PYG{o}{.}\PYG{n}{dnpResults}\PYG{o}{.}\PYG{n}{plt}\PYG{o}{.}\PYG{n}{figure}\PYG{p}{(}\PYG{p}{)}
\PYG{n}{dnp}\PYG{o}{.}\PYG{n}{dnpResults}\PYG{o}{.}\PYG{n}{plot}\PYG{p}{(}\PYG{n}{data}\PYG{p}{)}
\PYG{n}{dnp}\PYG{o}{.}\PYG{n}{dnpResults}\PYG{o}{.}\PYG{n}{plt}\PYG{o}{.}\PYG{n}{show}\PYG{p}{(}\PYG{p}{)}

\PYG{c+c1}{\PYGZsh{} Plotting a workspace (dnpdata\PYGZus{}collection)}
\PYG{n}{dnp}\PYG{o}{.}\PYG{n}{dnpResults}\PYG{o}{.}\PYG{n}{plt}\PYG{o}{.}\PYG{n}{figure}\PYG{p}{(}\PYG{p}{)}
\PYG{n}{dnp}\PYG{o}{.}\PYG{n}{dnpResults}\PYG{o}{.}\PYG{n}{plot}\PYG{p}{(}\PYG{n}{ws}\PYG{p}{[}\PYG{l+s+s1}{\PYGZsq{}}\PYG{l+s+s1}{proc}\PYG{l+s+s1}{\PYGZsq{}}\PYG{p}{]}\PYG{p}{)}
\PYG{n}{dnp}\PYG{o}{.}\PYG{n}{dnpResults}\PYG{o}{.}\PYG{n}{plt}\PYG{o}{.}\PYG{n}{show}\PYG{p}{(}\PYG{p}{)}

\PYG{c+c1}{\PYGZsh{} Plotting two curves on the same figure}
\PYG{n}{dnp}\PYG{o}{.}\PYG{n}{dnpResults}\PYG{o}{.}\PYG{n}{plt}\PYG{o}{.}\PYG{n}{figure}\PYG{p}{(}\PYG{p}{)}
\PYG{n}{dnp}\PYG{o}{.}\PYG{n}{dnpResults}\PYG{o}{.}\PYG{n}{plot}\PYG{p}{(}\PYG{n}{ws}\PYG{p}{[}\PYG{l+s+s1}{\PYGZsq{}}\PYG{l+s+s1}{proc1}\PYG{l+s+s1}{\PYGZsq{}}\PYG{p}{]}\PYG{p}{)}
\PYG{n}{dnp}\PYG{o}{.}\PYG{n}{dnpResults}\PYG{o}{.}\PYG{n}{plot}\PYG{p}{(}\PYG{n}{ws}\PYG{p}{[}\PYG{l+s+s1}{\PYGZsq{}}\PYG{l+s+s1}{proc2}\PYG{l+s+s1}{\PYGZsq{}}\PYG{p}{]}\PYG{p}{)}
\PYG{n}{dnp}\PYG{o}{.}\PYG{n}{dnpResults}\PYG{o}{.}\PYG{n}{plt}\PYG{o}{.}\PYG{n}{show}\PYG{p}{(}\PYG{p}{)}

\PYG{c+c1}{\PYGZsh{} Plotting with some custom parameters}
\PYG{n}{dnp}\PYG{o}{.}\PYG{n}{dnpResults}\PYG{o}{.}\PYG{n}{plt}\PYG{o}{.}\PYG{n}{figure}\PYG{p}{(}\PYG{p}{)}
\PYG{n}{dnp}\PYG{o}{.}\PYG{n}{dnpResults}\PYG{o}{.}\PYG{n}{plot}\PYG{p}{(}\PYG{n}{ws}\PYG{p}{[}\PYG{l+s+s1}{\PYGZsq{}}\PYG{l+s+s1}{proc}\PYG{l+s+s1}{\PYGZsq{}}\PYG{p}{]}\PYG{p}{,} \PYG{l+s+s1}{\PYGZsq{}}\PYG{l+s+s1}{k\PYGZhy{}}\PYG{l+s+s1}{\PYGZsq{}}\PYG{p}{,} \PYG{n}{linewidth} \PYG{o}{=} \PYG{l+m+mf}{3.0}\PYG{p}{,} \PYG{n}{alpha} \PYG{o}{=} \PYG{l+m+mf}{0.5}\PYG{p}{)}
\PYG{n}{dnp}\PYG{o}{.}\PYG{n}{dnpResults}\PYG{o}{.}\PYG{n}{plt}\PYG{o}{.}\PYG{n}{show}\PYG{p}{(}\PYG{p}{)}
\end{sphinxVerbatim}

\end{fulllineitems}



\section{hydrationGUI}
\label{\detokenize{hydrationGUI:hydrationgui}}\label{\detokenize{hydrationGUI::doc}}
Type hydrationGUI at the command line to open an interactive tool for processing ODNP data and calculating hydration parameters. All data processing and calculating is done using buttons, checkboxes, sliders, and edit fields.

Type the command to start the hydrationGUI:

\begin{sphinxVerbatim}[commandchars=\\\{\}]
\PYG{g+gp}{\PYGZgt{}} hydrationGUI
\end{sphinxVerbatim}

\begin{figure}[htbp]
\centering
\capstart

\noindent\sphinxincludegraphics[width=400\sphinxpxdimen]{{hydrationGUI_overview}.png}
\caption{hydrationGUI}\label{\detokenize{hydrationGUI:id1}}\end{figure}


\subsection{Processing a single topspin data folder, 1D spectrum or 2D inversion recovery data}
\label{\detokenize{hydrationGUI:processing-a-single-topspin-data-folder-1d-spectrum-or-2d-inversion-recovery-data}}
To work on a single topspin spectrum use the Bruker button to select a numbered folder containing a single spectrum, either 1D or 2D. You may make adjustments to the data phase and integration window center using the sliders. Use the “Optimize” checkboxes to search for and apply the “optimal” parameters.

\begin{figure}[htbp]
\centering
\capstart

\noindent\sphinxincludegraphics[width=400\sphinxpxdimen]{{hydrationGUI_importing_1d_2d}.png}
\caption{Importing 1d or 2d data}\label{\detokenize{hydrationGUI:id2}}\end{figure}

\begin{figure}[htbp]
\centering
\capstart

\noindent\sphinxincludegraphics[width=400\sphinxpxdimen]{{hydrationGUI_experiment_304}.png}
\caption{Processing T1 experiment}\label{\detokenize{hydrationGUI:id3}}\end{figure}


\subsection{Processing Han lab datasets}
\label{\detokenize{hydrationGUI:processing-han-lab-datasets}}
To load a dataset collected in the CNSI facility at University of California Santa Barbara using the ‘rb\_dnp1’ command, use the Han Lab button and select the base folder.

\begin{figure}[htbp]
\centering
\capstart

\noindent\sphinxincludegraphics[width=400\sphinxpxdimen]{{hydrationGUI_importing_rbdnp1}.png}
\caption{Importing "rb\_dnp1" experiment}\label{\detokenize{hydrationGUI:id4}}\end{figure}

The title of the main plot will let you know which folder you are currently working on. Use the Next button to advance through the dataset towards calculating hydration parameters, and the Back button to regress through the dataset. Auto Process will run through the entire dataset automatically and calculate hydration parameters.

\begin{figure}[htbp]
\centering
\capstart

\noindent\sphinxincludegraphics[width=400\sphinxpxdimen]{{hydrationGUI_procesing_rbdnp1_data}.png}
\caption{Advance through the individual datasets to process the data}\label{\detokenize{hydrationGUI:id5}}\end{figure}

You may make adjustments to the data phase, integration window width, and integration window center using the sliders. Use the “Optimize” checkboxes to search for and apply the “optimal” parameters. For optimizing the width, checking Optimize selects the window that encompasses roughly 2/3 of the peak while unchecking selects the default width. If processing an ODNP dataset the width that is displayed in the plot will be used if the Next or Auto Process buttons are pressed.

The results are displayed when finished. If a “Workup” is also present in the data folder it will be imported for comparison. Use the corresponding checkboxes to interact with the Workup results. Interaction with any parameter edit field or checkbox, as well as the T1 interpolation checkboxes, automatically updates the calculations.

The title of the main plot will let you know which folder you are currently working on. Use the Next button to advance through the dataset towards calculating hydration parameters, and the Back button to regress through the dataset. Auto Process will run through the entire dataset automatically and calculate hydration parameters.

\begin{figure}[htbp]
\centering
\capstart

\noindent\sphinxincludegraphics[width=400\sphinxpxdimen]{{hydrationGUI_ksigma}.png}
\caption{Hydration Results}\label{\detokenize{hydrationGUI:id6}}\end{figure}

The results are displayed when finished. If a “Workup” is also present in the data folder it will be imported for comparison. Use the corresponding checkboxes to interact with the Workup results. Interaction with any parameter edit field or checkbox, as well as the T1 interpolation checkboxes, automatically updates the calculations.

The Restart button will return you to the beginning of processing. If the Only T1(0) checkbox is selected, Restart will return you to the final folder that is the T1(0) measurement while all other processing will be retained. If the Only T1(p) is selected you will return to the beginning of the series of T1 measurements and previous processing of the enhancement points is retained.


\subsection{Analyzing previous GUI results Workup results}
\label{\detokenize{hydrationGUI:analyzing-previous-gui-results-workup-results}}
You may also load only the results of “Workup” code processing with the Workup button, or you may select the .mat or .h5 files of a previously saved session with the GUI Result button.

\begin{figure}[htbp]
\centering
\capstart

\noindent\sphinxincludegraphics[width=400\sphinxpxdimen]{{hydrationGUI_previous_results1}.png}
\caption{Hydration Results from workup}\label{\detokenize{hydrationGUI:id7}}\end{figure}

\begin{figure}[htbp]
\centering
\capstart

\noindent\sphinxincludegraphics[width=400\sphinxpxdimen]{{hydrationGUI_previous_results2}.png}
\caption{Hydration Results from h5}\label{\detokenize{hydrationGUI:id8}}\end{figure}

The results of previous processing will be used to calculate hydration parameters.

\begin{figure}[htbp]
\centering
\capstart

\noindent\sphinxincludegraphics[width=400\sphinxpxdimen]{{hydrationGUI_results_from_h5}.png}
\caption{Imported results from h5 file}\label{\detokenize{hydrationGUI:id9}}\end{figure}


\subsection{Terminal outputs}
\label{\detokenize{hydrationGUI:terminal-outputs}}
The terminal will display processing and calculation progress as well as standard deviations of the T1 fits and κσ, including the imported κσ if a Workup was found.

\begin{figure}[htbp]
\centering
\capstart

\noindent\sphinxincludegraphics[width=400\sphinxpxdimen]{{hydrationGUI_terminal}.png}
\caption{Terminal Output from processing}\label{\detokenize{hydrationGUI:id10}}\end{figure}


\subsection{Saving Results}
\label{\detokenize{hydrationGUI:saving-results}}
After processing is complete and hydration parameters are calculated, the Save results button is available. Your results are saved in .csv, .h5, and .mat formats. The .mat file can be read by the MATLAB app called xODNP that is available at MathWorks File Exchange. The .h5 and .mat files can be read by hydrationGUI.


\chapter{Index}
\label{\detokenize{index:index}}\begin{itemize}
\item {} 
\DUrole{xref,std,std-ref}{genindex}

\item {} 
\DUrole{xref,std,std-ref}{modindex}

\item {} 
\DUrole{xref,std,std-ref}{search}

\end{itemize}


\chapter{Acknowledgements}
\label{\detokenize{index:acknowledgements}}
Development of dnpLap is sponsored by grants by the United States and the Germany. In particular:
\begin{itemize}
\item {} 
 acknowledges support from the U.S. National Institutes of Health (NIH), grant .

\item {} 
The  acknowledges support from the U.S. National Science Foundation (NSF), Award No. CHE\sphinxhyphen{}1800596 and the German Science Foundation (DFG) as part of the Excellence Initiative under RESOLVDFG\sphinxhyphen{}EXC\sphinxhyphen{}2033 Project No. 390677874.

\item {} 
The  acknowledges the received Startup Funds by Syracuse University.

\end{itemize}


\renewcommand{\indexname}{Python Module Index}
\begin{sphinxtheindex}
\let\bigletter\sphinxstyleindexlettergroup
\bigletter{d}
\item\relax\sphinxstyleindexentry{dnpLab.dnpFit}\sphinxstyleindexpageref{dnpFit:\detokenize{module-dnpLab.dnpFit}}
\item\relax\sphinxstyleindexentry{dnpLab.dnpHydration}\sphinxstyleindexpageref{dnpHydration:\detokenize{module-dnpLab.dnpHydration}}
\item\relax\sphinxstyleindexentry{dnpLab.dnpImport.h5}\sphinxstyleindexpageref{dnpImport:\detokenize{module-dnpLab.dnpImport.h5}}
\item\relax\sphinxstyleindexentry{dnpLab.dnpImport.prospa}\sphinxstyleindexpageref{dnpImport:\detokenize{module-dnpLab.dnpImport.prospa}}
\item\relax\sphinxstyleindexentry{dnpLab.dnpImport.topspin}\sphinxstyleindexpageref{dnpImport:\detokenize{module-dnpLab.dnpImport.topspin}}
\item\relax\sphinxstyleindexentry{dnpLab.dnpImport.vnmrj}\sphinxstyleindexpageref{dnpImport:\detokenize{module-dnpLab.dnpImport.vnmrj}}
\item\relax\sphinxstyleindexentry{dnpLab.dnpNMR}\sphinxstyleindexpageref{dnpNMR:\detokenize{module-dnpLab.dnpNMR}}
\item\relax\sphinxstyleindexentry{dnpLab.dnpResults}\sphinxstyleindexpageref{dnpResults:\detokenize{module-dnpLab.dnpResults}}
\end{sphinxtheindex}

\renewcommand{\indexname}{Index}
\printindex
\end{document}